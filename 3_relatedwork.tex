\chapter{Related Work}
\label{ch:related}

In this chapter I discuss work that this thesis builds upon.  Numerous approaches exist for approximating temporal data and in the first section I discuss work
\section{Model-based Clustering}
\label{ch:related}
For the task of temporal clustering, the use of Markov-based approaches to model time series has been shown to produce high performance results in several domains.  The beneficial parametric assumptions they make are used for abstracting temporal data to describe temporal dependencies.  Almost all existing model-based methods use a Markov chain, many with extensions for hidden states, to represent the temporal dynamics of observation sequence.  Among the various techniques that appear in the research literature on Markov-model abstraction, the key distinction is how the model parameters are used for clustering.

The most popular approach is to describe a time series dataset as generated by $K$-HMM components and identify $K$.  For example, we can describe a dynamic process using:
      $$f_{K}(O)=\sum_{k=1}^{K}f_{k}(O|\theta_{k})w_{k}$$
where,  $O$ is a time series, $w_{k}$ is a weight of the $k$th HMM and $f_{k}(O|\theta{k})$is the probability that the series was generated by the component model $f_{k}$ with parameters $\theta_{k}$.  To estimate $K$ many clustering approaches of this type impose additional parametric assumptions about cluster shape~\cite{Shah2009,Zeng2006,preprints429}.

In terms of nonparametric approaches that complement modelbased abstraction, spectral methods have been paired with HMM abstraction for \emph{semiparametric clustering}.   In this approach, models are embedded in that their parameters are used as input to the clustering step and it is sometimes referred to as clustering with embedded models.   Here, the model represents the time series in a concise and descriptive way, and provides a principled approach to handling variable length temporal sequences~\cite{Yin_Yang_2005,Jebara_07}.  

 The semiparametric framework for time series clustering was motivated by the ambition to apply high-performance clustering methods that allow the data miner to be agnostic about the structure of the resulting group.   Notably, using this approach to semi-parametric clustering Jebara et al.~\cite{Jebara_07} have clustered motion capture data and show better performance relative to clustering with mixtures of HMMs, and comparable to that which was achieved with supervised learning.  Further building on the general semi-parametric framework, and also using embedded HMMs with spectral clustering, Garcia et al.~\cite{GarciaHD09,Garcia2011} examine the impact of alternative similarity metrics to compare models derived from individual observations and their ability to create structured distance matrices that are consequently exploited by spectral methods.

An alternative nonparametric approach for clustering time-series has been described in recent work using Markov model abstraction.  Fruhwirth et al.~\cite{Pamminger_Fr_2010} pose the clustering task as estimating the priors for $K$ components of a Dirichlet distribution.  Similar work in sequential data is more prevalent, and since a Bayesian method is typically used to estimate the conjugate priors, it is has been referred to as hierarchical Bayesian clustering.  Typically Markov chain Monte Carlo simulation techniques are used for parameter estimation, and existing work has demonstrated the approach for clustering gene expression data, sequential properties of natural language, and people by wage mobility~\cite{Pamminger_Fr_2010,TehJorBea2006,XinSohJor2006}.



\section{Continuous-time Bayesian Networks}
\section{Disease Progression Models}
Extensive work has been performed in igh-freq
