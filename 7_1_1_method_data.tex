\subsection{Data Description}
The hepatitis data set used for my experiments consists of blood inspection and urinalysis laboratory data that was provided by the Chiba University Hospital in Japan, and was used for the ECML/PKDD-2004 and 2005 Discovery Challenges.  It consists of test values for 771 patients with hepatitis type B or C, and spans the years 1982 through 2001.

For each patient, the data set includes demographics, the pathological classification of the disease, the date of each biopsy, the result of each biopsy, and blood test and urinalysis results.  For patients with hepatitis C, there is an additional indication for interferon therapy, which is used to treat the disease and can effect the values of indicators for hepatocyte inflammation.  Although the temporal data contain the results of 983 types of examinations, we use the key indicators that are noted in the literature, and those indicators that showed diagnostic relevance for the machine learning challenges.

At the time of the challenge, medical research reported platelet count values (PLT) were correlated with fibrosis score at the time of biopsy, but cross-sectional analysis provided limited predictive power.  It was hypothesized that analyzing the trend for each patient over time could provide additional information for categorizing hepatitis patients by disease-related risk types.  However, temporal analysis of the PLT data was rarely performed and limited by difficulties in time series comparison, irregular sampling intervals, and variable sequence lengths~\cite{Shoji05}.

There were several goals posed by the ECML/PKDD shared task.  The specific challenge that relates to this work was determining the value of longitudinal laboratory exam data for assessing liver fibrosis, and to better understand the temporal patterns that correspond with the results of biopsy grading and staging.  Specifically, \emph{are there temporal patterns that can be detected from lab data to help distinguish patients that progress to end stage liver disease and those that do not?}

Although what was mainly developed consisted of classifiers, clustering applications for this data set appear in the literature.  Most notably the work of X et al. in 2007, and I used their results as a benchmark to evaluate performance.  Using PLT, ChE and ALB lab tests, one system~\cite{Hirano05,Hirano07a, Hirano07b,Tsumoto12} demonstrated that medically relevant time series features associated with the progression of liver fibrosis could be learned from clustering methods.  Other lab tests that were reported by challenge participants as informative for predicting fibrosis stage included: zinc turbidity test (ZTT), albumin (ALB), bilirubin D-BIL and CHE(cite task overview paper). 