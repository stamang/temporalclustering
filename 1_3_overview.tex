\section{Overview}
This work extends applications of semiparametric clustering methods for temporal data and describes two experiments using clinical data sets to evaluate performance.

A background on temporal analysis methods, including semiparametric clustering is discussed in more detail in Chapter~\ref{ch:temporal_dynamics}.

Related work that this thesis builds upon in continuous-time representations, biostatistics, and Bayesian clustering is detailed in Chapter~\ref{ch:related}.


%Using electronic health record data from patients with or at risk of chronic disease as a case study, we develop a new approach for learning temporal dynamics from a secondary data source that contains variable length sequences, is measured at irregular time intervals, subject to arbitrary sampling schemes, and subject to incompleteness

%Related work that this thesis builds upon in continuous-time representations, epidemiology, and Bayesian clustering is detailed in Chapter~\ref{ch:related}.

 I describe the variety of methods for temporal abstraction in Chapter~\ref{ch:abstraction}.  Their relationship to clustering, and a comparison of clustering methods are provided in Chapter~\ref{ch:clustering}.

 My novel contributions begin in Chapter~\ref{ch:new}, where I discuss continuous-time representations for patient level disease modeling, the integration of nonparametric Bayesian clustering.

 I evaluate the performance of this new approach to semi-parametric clustering on two data patient-level data sets, including a hepatitis and glucose data sets in Chapter~\ref{ch:expts}.

Finally, a summary of this work, the key conclusions and next steps appear in Chapter~\ref{ch:conclusions}.
