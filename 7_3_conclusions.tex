\section{Summary}
The experimental methods and study results are presented for two distinct clinical data sets.  The first consists of patient lab data that was collected in Japan at in and outpatient visits, and represents individuals with HVB and HVC.  Our second data set was generated form an inpatient population at a large urban hospital with at least two indications of physician ordered glucose tests during their hospital stay.

The irregular sampling of clinical observations over variable durations is the main challenge for existing temporal mining methods.  The new techniques described in this thesis provide a principled method for their exploratory analysis.  Specifically, they can be used to preprocess patients with the aim of enriching the quality of the data by eliminating observations that are noisy or less relevant for the research purpose, or to discover patient groups that can help detail the natural history of population and subpopulation level disease phenotypes.

Although each study has the goal of reveling inherent patient groups that correspond with disease phenotypes, there are some distinctions between the two sets.  First, the hepatitis data set is more complete, combining both in-patient and out-patient laboratory examinations.  Also, the time of observation is less likely influenced by an informative sampling scheme. In the case of the inpatient glucose monitoring data, it is more likely that sampling time is influenced by self-selection sampling, and driven by the need to seek emergent care.

\section{Modeling Chronic Hepatitis}
 In our experiments, the dynamic patterns that are produced by modeling platelet count over time helps to group hepatitis B and C patients in a way that is useful for informing the prognosis and treatment of patients.  If used as a screening tool, our experiments suggest that for unseen patients that are most similar to our lowest risk cluster, it will likely result in unnecessary biopsies.

 However, for patients at higher risk, our method does not provide sufficient evidence to base prognosis and treatment on the basis of clustering assignments alone.  It does provide useful information for assessing higher risk patients, but associates some lower risk patient, increasing cluster heterogeneity.

 These findings are aligned with recent work in supervised learning~\cite{Jiang2006}, and distinct work in clinical research~\cite{Parkes2011} indicating methods that use noninvasive serological markers to assess fibrosis stage can reduce the number of unnecessary biopsies, or eliminate the need ~\cite{Gangadharan}.  Recent studies use numerous direct and indirect serum markers and even challenge the need for liver biopsy  or replace liver biopsy.  They do not attempt to model dynamic changes that are key to description  a patient trajectory, and instead entail test panels that provide a richer cross-sectional picture.

 For clustering temporal clinical data, the semiparametric Bayesian clustering explicitly models the dynamic changes that occur over the course of a patient's trajectory.  For chronic hepatitis patients, I identify key features for modeling patient disease dynamics by calculating the mutual information between lab tests that have Also, just more variables improve the performance of liver panels for assessing fibrosis, a more articulated abstraction model will also improve temporal clustering performance for identifying various risk strata.
