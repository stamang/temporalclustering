\chapter{Temporal Abstraction}
\label{ch:abstraction}

\emph{Abstraction} is the process by transforming a concept or an observable phenomenon in to a more succinct form, typically to retain only information which is relevant for a particular purpose.  For the purpose of inference, the goal of temporal abstraction is to provide a concise, high-level description of a sequential process that facilitates description and comparison among sequences, while simultaneously preserving the information contained in the raw data.

Complex data that has many observations over weeks, or even years, it not easily understood.  Even for trained experts, it can be hard to discern meaning from poorly structured, irregular data.  Data ming facilitates the analysis of data that is too large or complex for human review, and temporal mining focuses on data with an explicit temporal component.  In addition to facilitating the processing of data sets for knowledge discovery and mining, abstraction can benefit interpretation and comprehension to humans too.

Historically, the temporal dimension of data was mostly ignored due to the additional resources that processing or collection entailed.  Now, large-scale data analysis had become more efficient, and the amount of digital artifacts in the world is increasing at an exponential rate.  However, most existing longitudinal analysis techniques are designed for primary data sources, or require a canonical form of a time series.   Along with the many new data mining opportunities, are new limiting factors related to the analysis of noisy, incomplete secondary data sets that are repurposed as research tools.

\section{Representations}
Most all temporal analysis methods integrate an abstraction techniques.  Some researchers argue that it is the single most important step.  That is, if the quality of the approximation is high, the results that are achieved will be similar to what would be achieved using the raw data.  If abstraction is poor, dynamics are not sufficiently captured and can lead to false findings.

Even the ideal abstraction technique can lead to erroneous findings, so we do not make this case.  However, it is obvious that the choice of high-level representation is a key decision in any temporal mining framework, and especially important when data is irregularly sampled and noisy.  In this case, abstraction can help to mitigate the impact of noise, and transform uneven length observation lengths into a uniform format that allows for further processing.

One way to describe an abstraction approach is in terms of how a time series is input for a learning task:

\begin{description}
  \item[Raw Data] In general, raw data it not feasibly processed by clustering methods.  For small, structured, or synthetically derived data, clustering on the raw data sequence is possible.
  \item[Feature-based] Many feature extraction methods are simple in nature and help to characterize the shape of the trajectory that is monitored.  They aim to measure a quality about the observation sequence instead of the underlying data generation process and use the similarity among features for clustering.  Also, features tend to be domain, task and data set specific.  For example, a features that works well for one data set may be irrelevant to another application.  Some well-known approaches to feature extraction that have continued to be enhanced since their introduction are spectral techniques.  These are borrowed from signal-processing and include Discrete Fourier Transform (DFT)~\cite{HarrisF}, Discrete Wavelet Transforms (DWT)~\cite{Shensa}, and Piecewise Linear Approximation (PLA)~\cite{Dunham}.
  \item[Model-based] Working under the assumption that each time series was generated by an underlying process that can be modeled, the similarity between different models serves as the basis for time series clustering.  There have been various methods proposed for model-based abstraction in the recent literature and applications of probabilistic Graphical models (PGMs) for temporal modeling is an increasingly active active area of research~\cite{Jebara_07,Smyth97,Fruhwirth11}.  These models represent observable or partially observable problem elements as nodes and uses directed edges to indicate the probabilistic dependencies, or relationships, among nodes.  These dependences can be represented with one time slice to indicated conditional distributions associated with the process, and in the case of dynamic Bayesian networks (DBNs) between time slices in the direction of time.  One aspect of DBNs, a specific instance PGMs, that makes them particularly attractive for modeling sequential phenomena is the expressive language that can be applied to describe a process.  Also, that problem semantics and algorithmic components are distinct.  For example, when an analysis for a new data set is required, a DBN structure will need to be defined to represent the new problem's semantics, but all of the algorithms parts lend themselves to immediate reuse.
\end{description}

\subsection{Subsequence and Whole Sequence Approaches}

In addition to abstraction types, we can also describe an abstraction method in terms of the information represented in the approximated values.   In the past, work with subsequences was pervasive.  However, a study by Keogh et al. showed the limitations of subsequence-based approaches resulted into a shift towards whole sequence analysis.  Although there has since been work that addresses the shortcomings noted in this key paper, for many domains and applications these findings still hold relevance.
\begin{description}
  \item [Whole sequence]
  \item [Subsequence]
\end{description}

\section{Moving Average Models}

The most widely used family of techniques for modeling time are autoregressive techniques, known collectively as autoregressive moving average or ARMA models.  They make beneficial assumption to model the dependencies between adjacent observations time series, but don't directly represent problem semantics rather features of the time series.  The name comes from the combination of autoregressive (AR) features with moving average (MA) models to from ARMA(p,q) models, and use a predetermined, fixed size temporal window that slides along the entire duration of the time series.

 \emph{Autoregressive} (AR) models assume a current value depends on previous periods $p$ and white noise $\epsilon_t$.  For a model of order $p$, with model parameters $\phi$ at time $t$ $AR(p)$ is defined as:
 $$y_t= \sum_{i=1}^{p}\phi_{i}y_{t-i}+\epsilon_t$$

 A \emph{moving average} models of order $q, MA(q)$ with parameters $\theta$ and white noise $\epsilon_t$ is defined as:
 $$y_t= \epsilon_t+\sum_{i=1}^{q}\theta_{i}\epsilon_{t-i}$$

 ARMA models combine the two model types $AR(p)$ and $MA(q)$ and define $ARMA(p,q)$ by:

$$y_t= \epsilon_t + \sum_{i=1}^{p}\phi_{i}y_{t-i}+\sum_{i=1}^{q}\theta_{i}\epsilon_{t-i}$$

Strengths of ARMA techniques for time series modeling includes their ease of interpretation, prediction quality, and their ability to extend to the multivariate settings.  However, stationarity, or the assumption that the mean, variance and autocorrelation is uniform for the duration of $T$, is a beneficial property of autoregressive models that is not well matched with real-world problems that are often more complex.  Additional requirements of ARMA models are typically that the developer must have extensive knowledge and experience with the process that generated the data, and the technique requires a substantial amount of data preprocessing and parameter tuning.

\input{4_3_features}
\section{Graphical Models}
A popular type of model that provide useful parametric assumptions for modeling time are probabilistic graphical models (PGMs).  In addition to providing the potential for a succinct and natural representation of a temporal phenomena, PGMs can be constructed and utilized to provide new information while controlling computational costs. Unlike distance based metrics that have trouble extending to time series, they more naturally capture temporal correlations among neighboring states, and the importance of recent events relative to those in the distant past.

In addition to providing the potential for a succinct and natural representation of a temporal phenomena, PGMs can be constructed and utilized to provide new information while controlling computational costs~\cite{Jordan03gm}. Unlike distance based metrics that have trouble extending to time series, they more naturally capture temporal correlations among neighboring states, and the importance of recent events relative to those in the distant past~\cite{Fruhwirth-Schnatter11,hoppner:time,Jebara_07,koller}.


\subsection{Bayesian Networks}
Bayesian Networks (BNs) are a type of PGM that can be used to represent the key observable and latent variables of a system as \emph{nodes} in a graph and the probabilistic interactions between them as \emph{directed edges}.  In order to model a dynamic system where node values change dynamic Bayesian networks (DBNs) link $T$ template models, in this case BNs, with edges in the direction of time to define the conditional probability distribution for temporal transitions.

The graphical model formalism provides general algorithms for the calculation of marginal and conditional probabilities for a set of observations generated by various types of processes.  Also, they provide the flexibly for modeling complex problems.  Since each graph's semantics are separate from the algorithms that can be applied to the structures~\cite{koller} the algorithmic components can be reused more easily and have they have already demonstrated success for many types of learning tasks.


Bayesian Networks (BNs) are a type of PGM that can be used to represent the key observable and latent variables of a system as \emph{nodes} in a graph and the probabilistic interactions between them as \emph{directed edges}.  In order to model a dynamic system where node values change Dynamic Bayesian Networks (DBNs) link $T$ template models, in this case BNs, with edges in the direction of time to define the conditional probability distribution for temporal transitions. The graphical model formalism provides general algorithms for the calculation of marginal and conditional probabilities for a set of observations generated by various types of processes.  Also, they provide the flexibly for modeling complex problems, and each graph's semantics are separate from the algorithms that can be applied to the structures~\cite{koller}.  Since the algorithmic components can be reused more easily, and show good performance on very simple problem models, they have already demonstrated success for many types of supervised and unsupervised learning in across seemingly unrelated domains.

Our work focuses on modeling the temporal aspect of data, more specifically, data in the form of time series, and for that reason we limit further discussion to the use of graphical models in this context.  PGMs that formalize temporal dynamics typically represent a time series of length $T$ as a collection of ordered points points corresponding with a specific model state $Q$ that can assume one of $M$ possible values so that for any $t$, $Q_{t} \in \{1,...,M\}$.  We begin with a set of definition that are key to understanding how to construct an informative but concise abstraction of a system's temporal dynamics in a graphical structure.

\begin{description}
  \item[State Space Models (SSM)]  A SSM is a model that assumes an underlying state, or set of states, that are associated with a temporal phenomena and generates observations over time.  SSMs requires the definition of state transition probabilities, $P(Q_{t} | Q_{t-1})$, indicating the probability from transiting from one state to any other, the observation probabilities, $P(Y_{t} | Q_{t})$, indicating the probability of an observation given the current state value, and the initial state probability distributions $\pi$.
  \item[Markov property]  The Markov property states that the conditional probability distribution of a future state $Q_{t+1}$ is conditionally independent of states $Q_{i}$, where $i<t$, given $Q_{t}$.  Or more simply, given full knowledge of the current state, the future and the past are independent.  In general, if a simple but informative state description for a dynamic process can be defined, the Markov assumption is generally a reasonable approximation of the dependencies in a distribution.  A system for which the Markov assumption holds is considered a Markovian system, and PGMs that  relax the independence assumption are considered semi-Markovian.
  \item[Hidden Markov Model (HMM)] HMMs are use when model states are latent but for which correlated indicator variables exist. For HMMs, $Q_{t}$ is represented by a single `hidden' discrete random variable.  Figure~\ref{hidden} shows a HMM with 3 hidden states $Q_{1},Q_{2},Q_{3}$ that can assume any of the $M$ possible values and are correlated with the set of observations, $O_{1},O_{2},O_{3}$, also known as a Markov chain.  Like states, the values of observations come from a pre-defined set.  For the purpose of inference, these probability tables are known, or can be learned with optimization technique such as the forward-backward algorithm.   A more detailed description of HMMs are provided by Rabiner~\cite{Rabiner89atutorial}.

For the random variables in the temporal model, given the values of the variables in the previous model state the distribution of trajectories can be defined as the product of the conditional distributions and calculated using the chain rule.  More formally:
$$P(Q_{T})=\prod_{t=1}^{T}P(Q_{0})P(Q_{t+1} | Q_{t})$$

    \begin{figure}
\begin{center}
\begin{tikzpicture}[->,>=stealth',shorten >=1pt,auto,node distance=2cm,semithick]
\tikzstyle{every state}=[fill=blue!8,draw=black,thick,text=black,scale=1]


\node[state]         (A)              {$Q_{1}$};

\node[state]         (B) [right of=A] {$Q_{2}$};
\node[state]         (C) [right of=B] {$Q_{3}$};
\node[state,fill=blue!32]         (D) [below of=A] {$O_{1}$};
\node[state,fill=blue!32]          (E) [right of=D] {$O_{2}$};
\node[state,fill=blue!32]          (F) [right of=E] {$O_{3}$};

\path (A) edge  [right] node[below] {} (B);
\path (B) edge  [right] node[below] {} (C);
\path (A) edge  [below] node[below] {} (D);
\path (B) edge  [below] node[below] {} (E);
\path (C) edge  [below] node[below] {} (F);

\end{tikzpicture}
\end{center}
\caption{A hidden Markov model for time series of length $T=3$, where at any time $t \in \{1,...,T\}$ the value of a hidden model state is $Q_{t}$ and $O_{t}$ is the corresponding observation.}
\label{hidden}
\end{figure}
    %$$P(X^{0:T})=P(X^{0})\prod_{t=0}^{T-1} P(X^{t+1} | X^{0:t})$
  \item[Bayesian Networks (BNs)] A Bayesian Network, or Bayes net, is a probabilistic model that is represented as a DAG, where nodes represent random variables of interest, edges represent informational or causal dependencies among variables, and each node is conditionally independent of its non-descendants given the parents. Since they allow for a richer structural versatility, Bayes nets allow for more complex models to be represented.

      Let $X_i$ be a node in the BN, $X$, over the variables $X_{1},...,X_{n}$ and let Pa($X_i$) be the parents of $X_i$.  Based on the independence assumptions, we can define the joint probability distribution $P(X)$ using the chain rule for BNs as follows:
      $$P(X_{1},...,X_{n})= \prod_{i=1}^{n}P(X_i)|Pa(X_i)$$

    \item[Dynamic Bayesian Networks (DBNs)] For modeling time series, we can construct a DBN out of a BN, which functions as a \emph{template}, by instantiating the set of template variables for each time slice.  The set of template variables are repeated for each point in series and directed edges are added between time slices in the direction of time to reflect the template variables that interface between slices.  In a two-slice temporal Bayes net (2-TBN) edges are of two main types:  \emph{inter-time-slice edges} that interface between time slices or \emph{intra-time-slice-edges} that reflect the conditional dependencies among variables and within a single time slice.  Figure~\ref{patientdbn} shows a DBN with three instantiations of template variables.  Adjacent sets of template variables constitute a 2-TBN.
        \end{description}
        
\subsection{Markov Processes}
Markovian models are based on the underlying assumption that the future state of a system, $Q_{t+1}$, is independent of all past states, given the current state $Q_{t}$.  Hidden Markov models (HMMs) are useful for representing phenomena that care not directly observed, but for which a correlated variable can provide sufficient information to make inferences about the latent or \emph{hidden} state's value.

PGMs that formalize temporal dynamics typically represent a time series of length $T$ as a collection of ordered points points corresponding with a specific model state $Q$ that can assume one of $M$ possible values so that for any $t$, $Q_{t} \in \{1,...,M\}$.  One simplifying assumption for sequential data that has been widely applied for modeling many different types of sequential processes is the Markov assumption.  The Markov property states that the conditional probability distribution of a future state $Q_{t+1}$ is conditionally independent of states $Q_{i}$, where $i<t$, given $Q_{t}$.  Or more simply, given full knowledge of the current state, the future and the past are independent.  In general, if a simple but informative state description for a dynamic process can be defined, the Markov assumption is generally a reasonable approximation of the dependencies in a distribution.

\subsubsection{Hidden Markov Models}
HMMs are defined by a triple, $\{ A, B, \pi \}$ over a set of discrete states and distinct observations.  The matrix $A$ represents the state transitions, or the probability of moving from the current state, $q_{i}$, to the next state, $q_{i+1}$. The model parameter $B$ indicates the probability of an observation value for each $q\in Q$.

Although discrete-time models are suitable in many cases, there are two key limitations that have been noted (Saria 2007) and are directly relevant to the type of data typically found in provider databases.  First, if the underlying health related phenomena that is being modeled progresses in individuals at different rates, the smallest granularity must be used to express time steps for the entire system.  Second, when data is unavailable, intervening time slices must still be represented.


\subsubsection{Representation}
      We can view a HMM as a the simplest type of non-trivial DBN.  Alternative representation of DBNs extend the basic HMM model in that they are able to capture richer semantic information than HMMs, allowing temporal phenomena to be modeled more accurately.  For example, in a time slice DBNs allow for multiple model states; i.e., a DBN time slice $Z_{t}^{(i)}$ permits a set of $n$ states where $i \in \{1,...,n\}$.  Also, it permits a set of corresponding observations for $i$ that can assume discrete or continuous values.  Some other advantages over HMMs for modeling real-world phenomena is that they can relax the Markovian assumption to model semi-Markovian dynamics, and can be used to create multi-layer structures that have the potential to reflect local interactions among layers.

      The corresponding conditional distribution for the transition model of a 2-TBN makes two simplifying assumptions: (1) a dynamic system over the template variables $X$ satisfy the Markovian system and (2) a Markovian dynamic system is time invariant, or that $P(X_{t+1}|X)$ is the same for all values of $t$.  Since time is an infinite trajectory, these assumptions allow for modeling the state transition model more compactly.

      Let $Z$ be a set of $n$ persistent template variables that transfer temporal information in a 2-TBN.  We can represent the conditional distribution as:
      $$P(Z_{t+1} | Z_{t})= \prod_{i=1}^{n}P(Z_{t+1}^{(i)}|Pa(Z_{t+1}^{(i)}))$$
      \noindent For example, Figure~\ref{patientdbn} is a simple DBN model for monitoring a patient's disease acuity.  The variables are defined as : $A$=Access to care, $C$=Compliance, $D$=Disease Management, $R$=comoRbidities.  The model's structure conveys problem knowledge such as the level of compliance with treatment (C) is conditionally dependent on access to care (A).  Assuming all nodes are persistent, we can represent the conditional distribution based on the formula:
      $$P(Z_{t+1}|Z_{t})=P(Z_{t+1}^{A}|Z_{t}^{A})P(Z_{t+1}^{C}|Z_{t+1}^{A}Z_{t}^{C})P(Z_{t+1}^{D}|Z_{t+1}^{C}Z_{t}^{D}Z_{t+1}^{R})P(Z_{t+1}^{R}|Z_{t}^{R})$$
\begin{figure}[h]
\begin{center}
\begin{tikzpicture}[->,>=stealth',shorten >=1pt,auto,node distance=2cm,semithick]
\tikzstyle{every state}=[fill=blue!22,draw=black,thick,text=black,scale=1]

\node[state]         (A1)              {$Z_{1}^{A}$};
\node[state]         (C1) [below of=A1] {$Z_{1}^{C}$};
\node[state]         (R1) [right of=C1] {$Z_{1}^{R}$};
\node[state]         (D1) [below of=C1] {$Z_{1}^{D}$};


\node[state]         (C2) [right of=R1] {$Z_{2}^{C}$};
\node[state]         (A2) [above of=C2] {$Z_{2}^{A}$};
\node[state]         (R2) [right of=C2] {$Z_{2}^{R}$};
\node[state]         (D2) [below of=C2] {$Z_{2}^{D}$};

\node[state]         (C3) [right of=R2] {$Z_{3}^{C}$};
\node[state]         (A3) [above of=C3] {$Z_{3}^{A}$};
\node[state]         (R3) [right of=C3] {$Z_{3}^{R}$};
\node[state]         (D3) [below of=C3] {$Z_{3}^{D}$};

%\path (A) edge  [right] node[below] {} (B);
%\path (B) edge  [right] node[below] {} (C);
%\path (D) edge  [right] node[right] {} (B);
%\path (E) edge  [right] node[right] {} (C);

\path (R1) edge  [right] node[below] {} (D1);
\path (A1) edge  [right] node[below] {} (C1);
\path (C1) edge  [left] node[right] {} (D1);
\path (R2) edge  [right] node[below] {} (D2);
\path (A2) edge  [right] node[below] {} (C2);
\path (C2) edge  [left] node[right] {} (D2);
\path (R3) edge  [right] node[below] {} (D3);
\path (A3) edge  [right] node[below] {} (C3);
\path (C3) edge  [left] node[right] {} (D3);

\draw[dotted, ->] (5.5,-1) -- (6.5,-1) node [above]{$\Delta t$};
\draw[dotted, ->] (1.5,-1) -- (2.5,-1) node [above]{$\Delta t$};
\end{tikzpicture}
\end{center}
\caption{Dynamic Bayesian network (DBN), T=3}
\label{patientdbn}
\end{figure}
%\begin{figure}[!h]
%\centering
%\includegraphics[width=.85\linewidth]{fig/patientdbn.png}
%\caption{Dynamic Bayesian network (DBN)}
%\label{patientdbn}
%\end{figure}






