\chapter{Temporal Clustering}
\label{sec:clustering}
The goal of clustering algorithms is to partition $n$ observations from a population sample, $X$=$\{x_{1}, . . . , x_{n}\}$, so that each observation, $x_{i}$ where $i \in \{1,...,n\}$, is grouped into one of $k$ disjoint clusters.  They use a various optimization methods with the goal of assigning each $x_{i}$ to one group, where points in the same group are more similar and observations associated with different groups are dissimilar to each other.

The potential of clustering methods has already been demonstrated in the research literature.  Application have led to important findings in seemingly unstructured data.  For example, it has been successfully applied to identify regions in animal genomes that correspond with biomarkers for diseases such as cancer~\cite{Srivastava,Ewald} automatically identify themes in text \cite{Wang2012}, and many other learning tasks where labeled may be not be readily available~\cite{post08,XinSohJor2006}.

Despite successful applications, a topic that continues to be debated is the evaluation of clustering results~\cite{Guyon09}.  For many data sets, the search for a true or `gold' standard maybe futile.  When working with multi-faceted processes such as health, which can be assessed on the phenotypic and genotypic level, contradictions can appear when considering only one.  Similarly, classification of organisms also poses diverging opinions. Despite the hundreds of years of scientific examination by the most notable biologists in the world, and more recent innovations that allow for the sequencing of entire genomes, there is still arguments about the system of nomenclature.
%, leading us the the question, `If there is no true classification, how can there be a true clustering for a real-world problem'?

Although there has been a variety metrics for measuring the extrinsic and intrinsic quality of groups, some still consider clustering a craft rather than a science.  However, we continue to perform clustering, use these metrics, and compare results with established benchmarks and with good reason.  With the arrival of massively large social, and health data sets, many of which we predict are in their infant stages, methods that can be used to preprocess data, making consequent analytic steps or human perusal easier, or reveal unseen, meaningful pattern that were not predicted by researchers will be highly desired.

In terms of clustering the entities in our datasets, we make the assumption that \emph{a clustering assignment is valid if and only if the indicators for the phenomena within a group are correlated, but indicators for patients in different groups are uncorrelated or not as strongly correlated.}

Different clustering algorithms have been developed to suit a variety of tasks and types of data, and one way to broadly describe them is in terms of parametric assumptions.
Parametric clustering imposes a structure distribution to structures and this has clear benefits in terms of efficiency and performance when the assumption, e.g. clusters are Gaussian, is true.  In contrast, nonparametric clustering methods aim to be as agnostic as possible and infer the shape of clusters from the data and it is often used when no clear assumptions about the shape of clusters can be made.

Another important characteristic of clustering algorithms is the choice of similarity metric.  Since a good clustering is achieved when members of one cluster similar to each other and distinct from members of other clusters, how observation similarity is defined is a key aspect of a clustering method.  A common approach to evaluating similarities among a collection of observations is distance, and there are many techniques that are applied, including: Euclidian distance, Manhattan distance, and edit based distance.  Distance based similarity approaches are most effective when it can be reasonably assumed that the variables in each observation are independent and identically distributed (iid).  However, in the case of time-series data, where adjacent observations are more likely to be correlated with those in close proximity, and the dimension is unidirectional the development of accurate distance metrics can be more challenging.  For these reasons, techniques that can captures temporal dependencies in an earlier abstraction step or more accurately assess similarity have been an active area of research for a long time.

\section{Comparison of Techniques}
Some important aspect to consider when choosing a clustering algorithm for a new task are the compatibility of the data types in the dataset, the nature of underlying distributions, and the similarity measure.  Also, the more intuitive the approach and the easier it is to describe the characteristics of cluster properties, the more likely it is to translate into practice.

To identify the most appropriate clustering methods for the type of data that concerns this work, and described in more detail in X we assessed the potential of popular alternative clustering techniques that are well-represented in the literature, and several relatively unknown but promising approaches for our problem. We provide a summary of their benefits and limitations for clustering time series data in Table~\ref{clusteringAlgs}.


%The most popular clustering technique is $k$-means.  It is simple, efficient and performs well on many tasks.  Using unlabeled training set examples, the algorithm uses an iterative method to partition a data set of $n$ values into $k$ clusters without being told their categories. It begins by randomly assigning $k$ points as the initial `seed' representatives or centroids. After the initial assignment of $k$ points, an iterative process of reassignment based on the `closest' centroid, is repeated until a convergence criterion is met (e.g., the squared error ceases to decrease or their is no reassignment of centroid location) or until a pre-specified number of iterations is reached.

%The objective function of $k$-means is to minimize the total intra-cluster variance, and the common measure is the sum of the squared error:
%$$V = \sum_{k=1}^K \sum_{X\in C_k}||X- \mu_{k}||^2$$

%\noindent where $\mu_{k}$ is the center, or mean of cluster $C_k$, and where $||X- \mu_{k}||$ is the distance between a point in cluster $C_k$ and the cluster's centroid.  The default measure of distance is Euclidian distance, with ties broken arbitrarily.

%Limitations of $k$-means include sensitivity to initialization, faltering when the data set is not naturally represented as spherically shaped clusters and the presence of outliers, which can substantially influence the location of centroid centers.\newline

An important aspect to consider when choosing a clustering algorithm for a new task are the underlying distributions in the dataset.  If a common pattern relevant to the distribution of observations is known, then parametric assumptions about the shape of clusters can be exploited by an algorithms objective function to improve clustering performance.  If incorrect parametric assumptions are made, then resulting clustering assignments can be meaningless.  In the case when there is no evidence by which to assume parameters for the clustering model, nonparametric clustering is typically preferred.  Recent work has shown that these approaches are effective at clustering data without making what could be erroneous assumptions, and allow a data miner to be more agnostic about the shape of clusters.

We will use two approaches to nonparametric clustering.  The first approach will use model parameters for each time series as input to spectral methods that are described in more detail in Section X.  Variants of spectral clustering often differ in the representation of the graph Laplacian, and we use an approach that  normalizes the graph, and has been shown to enhance the clustering so that the cluster-properties in the data, so that groups in the data can be easily detected by $k$-means in high-dimensional space~\cite{Ng01onspectral}.

Our second clustering technique will use a two-level Bayesian hierarchy as described in Sectionwhere the base measure for a dataset is assumed to be a Dirichlet distribution.  Similarly to spectral clustering, hierarchical Bayesian methods do not impose strong structural assumptions about the shape of groups.  However, they are distinct in their approach to revealing clusters and offer some additional benefits in terms of interpretability.  For example, our medical dataset represents a population of patients at risk or with diabetes.  Therefore, the top level of the hierarchy provides information about the unknown heterogeneity that defines the variable dynamics within the population as a whole, and the first level provides information about unknown heterogeneity within individual groups.

For clustering real-world time series datasets, a two-level hierarchical Bayesian model offers several advantages:

\begin{enumerate}
\item a large number of real-world datasets provide information about a population based phenomena for which some prior information about group variability is known,
\item certainty about prior information is explicitly provided to the estimation procedure,
\item it allows us to interpret results in terms of population based and group based heterogeneity, and
\item when the size of the dataset is small, it is less prone to overfitting than other approaches.
\end{enumerate}

Model based Bayesian inference provides a framework for approximating priors.  In the case of model-based temporal abstraction, we estimate priors for the model parameters instead of the raw data.  Previous work has shown the appropriateness of this general approach for modeling natural language and the detection of biomarkers, and we will use similar inference techniques based on Gibbs sampling, a popular Markov Chain Monte Carlo algorithm.

\subsection{Spectral Methods}
\label{subsec:sc}
\emph{Spectral graph theory} has been applied in many fields to address the limitations posed by centroid-based definitions of a cluster.  They are based on performing the eigenanalysis of graphs.  \emph{Spectral clustering} recasts graph partitioning as a eigenvector problem.

Eigen- is a German word for "self" or "characteristic".  Eigenvalues are derived from a $n$x$n$ (square) matrix and typically represented as $\lambda$.  If $Ax = \lambda x$ and $x$, where $A$ is a $n$x$n$ matrix and $x$ is a non-zero vector, then $x$ is the eigenvector.

Eigenvectors are also known as `steady-state' vectors.   For example, if a Markov chain converges after many steps any row of that matrix is an eigenvector for $A^T$.   Eigenvectors are useful for characterizing the motion structure, and the one of highest magnitude, the `dominant eigenvector', can be described as a natural frequency.

Spectral clustering provides a non-parametric approach by using eigenvector segmentation, or graph partitioning based on graph cuts.  Matrix theory allows for the rewriting of a matrix in terms of smaller matrices, or blocks.  In the context of clustering, the blocks correspond to bunches, or groups of points where the similarity among points in the same bunch is high, but low relative to other points in alternative bunches.

Using a graph, $G=(V,E)$ where $V$ is a set of observations represented as vertices $\{x_{i}, . . . , x_{n}\}$ and $E$ a set of edges representing the similarity between observations, spectral clustering algorithms formalize the partitioning, problem with a variety of different approaches~\cite{Shi00,Ng01onspectral}.  There is no clear best method, but what is common to all of them is the use of an $nxn$ matrix to stores values to indicating the strength of the relation, $x_{i,j}$ where $i$ and $j\in \{1,...,n\}$, weighted by similarity.  The more similar $x_{i}$ and $x_{j}$ the higher the value.  In order to make computation more efficient, for each point only the number of nearest neighbors, $n$, appear in $S$.

Given the set of observations to be clustered, $X=\{x_{1}, . . . , x_{n}\}$ and the number of clusters $k$, a general approach to spectral clustering is:

\begin{itemize}
	\item Represent the similarity among all pairs of points as an $n$x$n$ affinity matrix, $W$, where $W_{ii}=0$
	\item Compute the Laplacian matrix, $L$ from $W$
	\item Calculate the top $k$ eigenvectors of $L$
	\item Define $U$, as an $n$x$k$ matrix using the $k$ eigenvectors of $L$
     \item Apply $k$-means on the rows of $U$ to obtain a clustering assignment $C=\{c_1,...,c_k\}$
     \item Assign each $x_i \in X$; if row $i$ of the matrix $U$ was assigned to the cluster $C_j$ where $j \in \{1,...,k\}$ then $x_i$ is a member of $C_j$
 \end{itemize}

Figure~\ref{min_cut} visualizes the nodes and edges connecting nearest neighbors.  To calculate the similarity between any two observations, the pairwise affinity, $w_{i,j}$, is computed using the norm of the difference between the two vectors $x_{i}$ and $x_{j}$:
$$w_{ij}=d(x_{i},x_{j})=\text{exp}\left\{\frac{||x_{i}-x_{j}||}{\sigma^{2}}\right\}$$
where the parameter $\sigma$ controls the width of local neighborhoods in the data.

\begin{figure}[h]
\centering
\includegraphics{fig/min_cut.jpg}
\caption{Minimum Cut}
\label{min_cut}
\end{figure}

The weighted adjacency matrix of $G$ is the matrix $W = (w_{ij})i,j=1,...,n$ representing the weights between all connected points, or `affinity'. If $w_{ij} = 0$, then $x_{i}$ and $x_{j}$ are not connected by an edge.

The minimum cut of the the weighted adjacency matrix, $W$, determines the optimal partitioning of the dataset.  A cut between any two vertices can be calculated as follows:
$$Cut(C_{1},C_{2})=\sum_{i\in C_{1}}\sum_{i\in C_{2}}w_{ij}$$

Spectral clustering algorithms recursively partitions a dataset by identifying the minimum cut and removes edges until $k$ clusters are identified. The problem of identifying the minimum cut is NP-hard; however there are more efficient approximations that are based on linear algebra using graph Laplacians and their basic properties.

The degree matrix $D$, is defined as the diagonal matrix of the degrees $d_{1},...,d_{n}$, where each degree of a vertex $x_{i} \in V$ is determined by the sum of the weights for the row:
$$d_{i}=\sum_{j=1}{n}w_{ij}$$

The graph Laplacian is defined using the degree matrix, $D$, and the $W$:
$$L=D-W$$
\noindent and used to identify the first $k$ eiginvectors that are used to create an eigenvector transformation of the data $U$.  Operating on $U$, $k$-means is then used for clustering the data and in the final step in the projected back to the initial data representation.

\subsection{Nonparametric Bayesian Clustering}
Bayesian nonparametric models have applied for both supervised and unsupervised learning task where it's desirable for the number of modeling parameters to adapt with the complexity of the data.   For this reason, they are labeled `nonparametric' and contrast parametric methods that require models parameters are fixed.  Often named by the processes they are used to model, they can be used for clustering, as a density estimator, the features for regression, and more.

The models assume an infinite parameter space, but use only a subset of all potential parameters to define a finite data set.   Since the parameters are a function the finite data set, or observation sample, the parameters that explain the underling process can vary with the size and complexity of the data (citation Peter Orbanz).

A main challenge posed by many traditional clustering algorithms is selecting the number of groups, $k$.   Some approaches uses to estimate $k$ are based on the spectral gap, or predictive estimates.  However, these are heuristics, and don't guarantee the choice of $k$, and more importantly, for many clustering problems $k$ is unrealistic to assume that $k$ is fixed.

By defining the clustering problem as identifying the components of infinite mixture, where $k$ is random variable in the model,  nonparametric Bayesian approaches allow for the definition of more flexible clustering models.  Relative to other nonparametric clustering methods such as spectral clustering, they do not require that $k$ is expressed a priori.  Also, nonparametric Bayesian cluttering provides a generative model that can describe group structure at the population and subpopulations level, more easily lending itself to interpretation by a domain expert.

\subsubsection{Mixture Models}
nonparametric Bayesian methods can be used to identify the number of component, and their densities, in a finite mixture model.  The density function of a finite mixture model is defined as:

$$p(x) = \Sigma_{k=1}^K \pi_k p(x|\theta_k)$$

where $x$ is the data set, $\pi$ is the mixing proportion, and $\theta_k$ are the model parameters for the cluster $k$.

In the nonparametric Bayesian application setting, we define the mixture model as that of one with infinite components.  We can define the discrete case in the form of the integral $p(x) = \int p(x|\theta)G(\theta)d\theta$, where $G = \Sigma_{k=1}^K \pi_k \delta_{\theta_k}$~\cite{OrbanzT10}, which extends to the following in the case of infinite components:

$$G = \Sigma_{k=1}^\infty \pi_k \delta_{\theta_k}$$

for a Bayesian non-parametric models with a potentially infinite value of $k$.

\subsubsection{Dirichlet Distribution}
In Bayesian statistics, a \emph{Dirichlet distribution}, Dir$(\alpha)$, is the conjugate prior of the categorical distribution and multinomial distribution.   In terms of a finite mixture model with $k$ components, it is a $k$-dimensional generalization of the beta distribution.

Since we can view model components as groups in the data, it has natural extensions for clustering.   Specifically, if a population can be described by the probability distribution $\Theta$ with components $\theta_{1},...,\theta_{k}$ that sum to 1, we can reasonably infer

$$\Theta \sim \text{Dirichlet}\{\alpha_{1},...,\alpha_{k}\}$$

The probability density function for a Dirichlet distribution uses a normalization factor that is defined in terms of the multinomial beta function, $B(\alpha)$,  that is expressed in terms of the gamma function:

$$B(\alpha)= \frac{\prod_{i=1}^{k} \Gamma(\alpha_{i})} {\Gamma(\sum_{i=1}^{k} \alpha_{i})}$$

\noindent and the probabilities $p$ and parameters $\alpha$ of each of the $k$ components:

$$\text{Dirichlet}(p;\alpha)=\frac{1}{B(\alpha)}\prod_{i=1}^{k} p_{i}^{\alpha_{i}-1}$$

\subsubsection{Dirichlet Process Gaussian Mixture Modeling}
One approach to nonparametric Bayesian clustering is Dirichlet Process Gaussian Mixture Modeling (DPGMM).   A Bayesian approach requires that a prior distribution is assigned to a model, and the uncertainly in the parametric form can be expressed as a Dirichlet prior~\cite{Gorur}.

Here, a Dirichlet process (DP) is the prior over the mixing distribution, $G$.  The DP has several algorithmic metaphors that help to describe approaches to the specification of the prior parameters.  Using exchangeability, the Polya urn problem can be used to describe a DP.  Another popular scheme is the the Chinese restaurant process (CRP), which describes a distribution of groups.  Lastly, the representation of the DP can be described as a stick-breaking prior.

A Dirichlet process defines a distribution over distributions. Also, referred to as a measure on measures it is characterized by two parameters: a base distribution $G_0$, from which samples are drawn, and a positive scaling parameter $\alpha$

$$G \sim DP(G_0,\alpha)$$

More intuitively described as a `splitting' criteria, $\alpha$ is a scaling factor that is associated with the probability of forming a new cluster.  The base distribution is defined by $G_0$.

For a sample, $G$, drawn from the base distribution $G_0$, if $G \sim DP(G_0,\alpha)$ then for any set of partitions $A_1 \cup A_2 \cup ... A_k$ of $A$:
$$(G(A_1),...,G(A_k)) \sim Dir(\alpha G_0(A_1),...,\alpha G_0(A_k))$$

(cite fergusen 1973 somewhere).

In the the Dirichlet process mixture model, the DP is used as nonparametric prior in a hierarchical Bayesian model, where $G$ is portioned according to the prior.   It was first applied for density estimation, but is now applied widely for clustering.   Dirichlet Process Gaussian Mixture Modeling (DPGMM) defines a  Dirichlet process mixture model by taking the limit of the number of mixture components as a hierarchical Gaussian mixture model approaches infinity.  Two methods used to specify the priors are Markov cain Monte-Carlo (MCMC) and variational inference.







\section{Evaluating Cluster Quality}
\label{sec:clusterEval}
Evaluating the results of clustering is a challenging task.  Although there has been exciting theoretical work in clustering, and many application demonstrate meaningful results, there is an need to establish a better and deeper science than is currently offered to address the issues that are independent of specific clustering methods~\cite{Pelillo,Guyon,Blum}.

For the purpose of evaluating clustering algorithms, we assess \emph{intrinsic} and \emph{extrinsic} cluster quality.  Metrics used in practice enable these judgements to be made in terms of abstract properties that exist independently of the data set.  However, these metrics make assumptions about conceptual questions such as ``what is a \emph{optimal} clustering'', which may require domain knowledge.  Also, research shows that humans employ multiple strategies for finding the number of clusters, $k$, and even on simple data sets the number of possible interpretations can be high~\cite{Lewis09}.

Despite their short-comings, validation metrics are useful for developing systems and evaluating clustering results, and researchers continue use and improve them.  A cluster can be informally described as a maximally coherent subset, $C$, that satisfies both inter and intra-cluster criterion:
\begin{itemize}
  \item items in $C$ should be homogeneous in type
  \item no larger cluster should contain $C$ as a proper subset
\end{itemize}
\noindent These generic qualities of an optimal clustering are generally agreed upon, and serve as the basis for developing metrics, such as purity, B-cubed, V-measure and others.

For simple data sets, evaluation is more straight-forward.  However, for more complex tasks some challenges that are in direct conflict with evaluating clustering results based on established metrics alone are:
\begin{itemize}
  \item for the same set of observations, an optimal clustering cannot be established
  \item the relative importance of clusters may be unequal and dependent on problem context; i.e. maximal coherency may be more important for certain clusters, and
  \item categorical similarity can be non-metric.
\end{itemize}

\subsection{Intrinsic Validation}
When a gold standard is unavailable to evaluate clustering performance, heuristics can be used to assess the \emph{intrinsic quality} of clusters.  

\subsubsection{Silhouettes}
One common heuristic that helps to quantify the intrinsic goodness and visualize cluster differences is the silhouette method~\cite{Rousseeuw}.  For a clustering assignment, the data set's silhouette is defined by the difference of the average dissimilarity of a point to members or its own cluster with that of the `neighboring' cluster over the max of these two dissimilarity measures.

For example, the silhouette validation technique can be used to validate patient clusters when human judgement is not available. For each patient, let $a(i)$ be the average dissimilarity of patient $i$ to all patients in its respective cluster.  We then calculate the average dissimilarity of patient $i$ with patients of another cluster, repeating for all clusters that patient $i$ is not a member of.  The cluster with the lowest average dissimilarity is the ``neighboring cluster'' of patient $i$ and indicated by $b(i)$. The resulting silhouette value is defined as:
$$s(i) = \frac{(b(i)-a(i))}{\mbox{max }{(a(i),b(i))}}$$
and the average $s(i)$ of a clustering assignment is a measure of how tightly patients are grouped into their respective clusters and how distinct clusters are with respect to each other.

When $s(i) \geq .6$, patient $i$ can be considered to be appropriately clustered.  A value close to -1 indicates that a patient would have been more appropriately assigned to the neighboring cluster, $b(i)$, and a value close to 1 indicates that individuals in the patient's respective cluster are very similar and that the cluster is distinct from other clusters.

\subsection{Extrinsic Validation}
 A clustering assignment that demonstrates high intrinsic quality may not always translate to high \emph{extrinsic quality}.  When available, comparison with a gold-standard is preferable.

Amigo et al.~\cite{Amigo} define formal constraints extrinsic metrics should satisfy based on the generic properties of, homogeneity, completeness, and cluster size vs. quantity.  A fourth, coined the `rag bag', places preference on a clustering that places `miscellaneous' items together, forming a set containing a `diverse genre'.  Although this constraint is not grounded in the generic qualities of a optimal clustering, it appears to facilitate the interpretation of clustering results, and has been demonstrated as a beneficial property in the researcher's human subjects study, which is used to validate the proposed constraints.

Using flow cytometry data, Aghaeepour et al. ~\cite{Aghaeepour} provide an empirical study of cluster evaluation metrics.  Their work indicates a correlation between the main types of metrics, and compares the seven different metrics to provide indications for the best metrics for clustering solutions against ground truth partitions.

\subsubsection{F-measure and B-cubed}
 B-cubed~\cite{Bagga} is a common measure for extrinsic evaluation that extends the F-measure to clustering ad decomposes precision and recall to each item.  This metric satisfies all of the constraints formalized by Amigo et. al and validated by human assessment~\cite{Amigo}, and F-measure alone reported the lowest overall error for all samples in the empirical study by Aghaeepour et al. ~\cite{Aghaeepour}.
 
The recall and precision for class $i$ in cluster $j$ can be expressed as:
$$R(c_{i},k_{j})=\frac{n_{ij}}{|c_{i}|}$$
$$P(c_{i},k_{j})=\frac{n_{ij}}{|k_{i}|}$$
\noindent where ${|c_{i}|}$ and ${|k_{j}|}$ represent the number of points in class $i$ and in cluster $j$ respectively.

Thus, B-cubed is defined as an \emph{element level precision and recall} as follows:
$$P=\frac{\sum{i}P(e)}{n}$$
$$R=\frac{\sum{i}R(e)}{n}$$

\subsubsection{Entropy-based Metrics}
The V-measure~\cite{Rosenberg} is a conditional entropy-based measure that addresses some of the limitations of other entropy-based metrics.  It satisfies all but the `rag bag' constraint discussed by Amigo et al.~\cite{Amigo}, and is the best performing entropy-based cluster evaluation metric reported in the empirical study by Aghaeepour et al. ~\cite{Aghaeepour}.  The metric explicitly measures how well a clustering application has achieved the qualities of homogeneity and completeness, by taking the harmonic mean of these scores, and like the F-measure can also be weighted.  

For a set of $N$ observations, a set of $n$ classes, $C=\{c_i|=1,...,n\}$ and $m$ clusters, $K=\{k_i|=1,...,m\}$, let $A$ be a contingency table with entries for all where $a_{ij}$ is the number of data points in both class $c_i$ and cluster $k_j$.  The homogeneity criteria is maximized when each cluster contains members of only one class is calculated by
$$h = 1-\frac{H(C|K)}{H(C)}$$
\noindent where $H(C|K)$ is the conditional entropy
$$H(C|K)=-\sum_{k=1}^m \sum_{c=1}^n \frac{a_{ck}}{N}\text{log} \frac{a_{ck}}{\sum_{c=1}^n {a_{ck}}}$$
\noindent and the class entropy $H(C)$ is given by
$$H(C)=-\sum_{c=1}^n \frac{a_c}{N}\text{log} \frac{a_c}{N}$$
\noindent Completeness is calculated by examining the distribution of the cluster assignments in each class
$$c = 1-\frac{H(K|C)}{H(K)}$$
\noindent with $H(K|C)$ and $H(K)$ in a symmetric manor.

The V-measure takes the harmonic mean of the homogeneity and completeness, and similar to the F-measure can be weighted to favor one criteria:
$$V_\beta = \frac{(1+\beta)*h*c}{(\beta*h)+c}$$ 

\subsection{Importance of Problem Context}
Clustering encompasses a wide range of problem types, many of which can be expressed in a taxonomy to help clarify design decisions and evaluation of the results~\cite{Guyon,Blum}.  However, for evaluating the goodness of clusters, existing metrics are calculated independent from problem context and have generally been developed to evaluate solutions for techniques that recast clustering as a partitioning problem, where the number of clusters are known $a priori$.

In contrast, human interpretation of clustering assignments takes into account the nature of the relations between how data representation and the clustering goals.  For this reason, it's treatment as an application-independent mathematical problem with associated metrics is limiting.  Therefore, in addition to validation metrics, interpreting clustering results in the context of problem semantics is also an important part of determine if a clustering application has produced meaningful results.

