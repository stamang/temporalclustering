\section{Application: Grading and Staging Liver Disease}
\subsection{Background}
`Hepatitis' is a Greek word with the root `hepat' meaning liver and the suffix `itis' indicating inflammation.  It is used to describe a class of viruses that are associated with liver inflammation and is characterized by three main types, A, B and C.  Hepatitis A is associated with acute inflammation of hepatocytes (`cyte' is a suffix meaning cell), and types B and C, chronic inflammation.  Individuals with types B and C have the potential risk of developing more severe stages of the disease that may result in permanent scaring of liver tissue, a condition known as cirrhosis, and hepatocarcinoma, a common type of liver cancer.

An indicter of cirrhosis or hepatocarcinoma is fibrosis of the hepatocytes.  A liver biopsy is the gold standard for the prognosis and treatment of hepatitis B and C and provides information that is used monitor disease progression and inform the treatment of hepatitis B and C patients.     Its primary purpose is to determine the health of the liver by measuring the degree of inflammation and stage of fibrosis.  Two common assessment tools are the Metavir test are shown in Table~\ref{metavir} and the Histologic Activity Index (HAI) shown in Table~\ref{hai}.  The Matavir score gives an indication of the amount of scarring, or fibrosis.  HAI represents an activity level, that corresponds with hepatocyte inflammation.

\begin{table}[ht]
\caption{Metavir Fibrosis Scoring}
\label{metavir}
\vskip 0.15in
\begin{center}
\begin{tabular}{p{0.05\linewidth} p{0.6\linewidth}}
\hline
\textbf{Score}	&  \textbf{Assessment} \\
\hline
0         & no scarring\\
1         & minimal scarring\\
2	     & scarring has occurred and extends outside
 the areas in the liver that contains blood vessels\\
3	     & bridging fibrosis is spreading and
 connecting to other areas that contain fibrosis \\
4	     & cirrhosis or advanced scarring of the liver\\
\hline
\end{tabular}
\end{center}
\vskip -0.1in
\end{table}

\begin{table}[ht]
\caption{Histologic Activity Index}
\label{hai}
\vskip 0.15in
\begin{center}
\begin{tabular}{ll}
\hline
\textbf{Score}	&  \textbf{Assessment}\\
\hline
0 & no inflammation  \\
1-4 & minimal inflammation  \\
5-8 & mild inflammation  \\
9-12 & moderate inflammation  \\
13-18 & marked inflammation  \\
\hline
\end{tabular}
\end{center}
\vskip -0.1in
\end{table}

Although it provides important clinical information to guide the care of chronic disease patients, biopsy is invasive, costly and subject to complications.  Biopsy involves extracting a tissue sample of at least $2-3$ cm in length with a 16-gauge needle that is inserted between two of the patients ribs.  The procedure has been association with complications that can be potentially life-threatening.  Also, can cost hundreds of dollars, and is subject to diagnostic error~\cite{Castera2012}.   For these reasons, alternatives for assessing the stage of liver fibrosis are in great demand and the lack of alternative assessment methods has been noted as a major limitation in both management and research in liver diseases.

\subsubsection{Temporal Indicators}
The progression from chronic hepatitis to hepatocarcinoma via liver cirrhosis is mainly observed in hepatitis C patients.  However, the detailed mechanism of this disease trajectory is not well-understood.  Recent work (cite) has demonstrated that temporal patterns associated with hepatitis indicators can provide useful information about liver activity for predicting fibrosis and present a viable option to more invasive testing.  These indicators are recorded as part of a standard liver panel, a collection of tests that are preformed together and provide useful information to assess liver damage, or urinary tests.

\subsection{Data Description}
The hepatitis data set used for my experiments consists of blood inspection and urinalysis laboratory data that was provided by the Chiba University Hospital in Japan, and was used for the ECML/PKDD-2004 and 2005 Discovery Challenges.  It consists of test values for 771 patients with hepatitis type B or C, and spans the years 1982 through 2001.

For each patient, the data set includes demographics, the pathological classification of the disease, the date of each biopsy, the result of each biopsy, and blood test and urinalysis results.  For patients with hepatitis C, there is an additional indication for interferon therapy, which is used to treat the disease and can effect the values of indicators for hepatocyte inflammation.  Although the temporal data contain the results of 983 types of examinations, we use the key indicators that are noted in the literature, and those indicators that showed diagnostic relevance for the machine learning challenges.

At the time of the challenge, medical research reported platelet count values (PLT) were correlated with fibrosis score at the time of biopsy, but cross-sectional analysis provided limited predictive power.  It was hypothesized that analyzing the trend for each patient over time could provide additional information for categorizing hepatitis patients by disease-related risk types.  However, temporal analysis of the PLT data was rarely performed and limited by difficulties in time series comparison, irregular sampling intervals, and variable sequence lengths~\cite{Shoji05}.

There were several goals posed by the ECML/PKDD shared task.  The specific challenge that relates to this work was determining the value of longitudinal laboratory exam data for assessing liver fibrosis, and to better understand the temporal patterns that correspond with the results of biopsy grading and staging.  Specifically, \emph{are there temporal patterns that can be detected from lab data to help distinguish patients that progress to end stage liver disease and those that do not?}

Although what was mainly developed consisted of classifiers, clustering applications for this data set appear in the literature.  Most notably the work of X et al. in 2007, and I used their results as a benchmark to evaluate performance.  Using PLT, ChE and ALB lab tests, one system~\cite{Hirano05,Hirano07a, Hirano07b,Tsumoto12} demonstrated that medically relevant time series features associated with the progression of liver fibrosis could be learned from clustering methods.  Other lab tests that were reported by challenge participants as informative for predicting fibrosis stage included: zinc turbidity test (ZTT), albumin (ALB), bilirubin D-BIL and CHE(cite task overview paper). 

\subsection{Abstraction}
\subsubsection{CTBN Model Structure}
A CTBN was used to abstract temporal information and is shown in Figure~\ref{msmgraph}.  The lab results for each patient over time were mapped to corresponding \emph{low}(L), \emph{normal}(N) and \emph{high}(H) values based on thresholds identified by clinical experts, and used as states in the model.  Table~\ref{hepxtab} shows the probability of each transition type based on the frequencies all patient transitions for platelet test values.  For example, based on reported transitions, patients reporting a high platelet count show the following probabilities for their next next observation period: high 92\%, normal 83\% and low is less than 1\%.
\emph{can we provide that averages for all patients here?}

\begin{figure}
\begin{center}
\begin{tikzpicture}[->,>=stealth',shorten >=1pt,auto,node distance=4cm,
  thick,main node/.style={circle,fill=blue!20,draw,font=\sffamily\Large\bfseries}]

  \node[main node] (1) {L};
  \node[main node] (2) [below left of=1] {N};
  \node[main node] (4) [below right of=1] {H};

  \path
    (1) edge node [left] {$q_{L,H}$} (4)
        edge [bend right] node[left] {$q_{L,N}$} (2)
    (2) edge node [right] {$q_{N,L}$} (1)
    	edge [bend right] node[below] {$q_{N,H}$} (4)
    (4) edge node [above] {$q_{H,N}$} (2)
        edge [bend right] node[right] {$q_{H,L}$} (1);
\end{tikzpicture}
\end{center}
\caption{3-state MSM}
\label{msmgraph}
\end{figure}

\begin{table}[ht]
\caption{Input for model abstraction step}
\label{hepxtab}
\vskip 0.15in
\begin{center}
\begin{tabular}{ l | c | c |c}
 $q_{i,j}$      & H & N & L  \\
\hline
H  & 0.915 & 0.083 & 0.001  \\\hline
N  & 0.035 &0.961 &0.004 \\\hline
L   & 0.025 &0.224 &0.751  \\
\end{tabular}
\end{center}
\vskip -0.1in
\end{table}

\subsubsection{Model Variables}
Based on an initial run using six lab tests that were selected due to known associations with liver decline, three, the PLT, ALB, and ZZT tests, resulted in good cluster assignments, with PLT performing the best based on the b-cubed metric~\cite{Bagga}, which is discussed in more detail in Section X, and is the harmonic mean of the average pointwise precision and recall for the cluster assignments compared with a gold standard.

To compare our results with previously published results I cluster patients based on PLT data alone, and in the multivariate lab test data.  To select additional indicators that are most useful, we generated preliminary clustering results using the five indicators that had been reported to have predictive qualities by unsupervised or supervised learning methods.  Using the variables that produced comparable results with that of PLT only, we calculated the mutual information between clusters assignments, and excluded ChE and D-BIL on this basis.  For multivariate temporal clustering of temporal ALB and PLT  data imporved results were achieved compared with results based on PLT alone, and ZZT made a minor improvement upon the results of ALB-PLT based clustering.

\subsubsection{Learning}
Multiple variables are combined for clustering, but the temporal data for each is abstracted separately.  Although the language of CTBNs allows for covariances, they must appear be observed concurrently, and there are numerous time days where the data does not appear for all three.  For example, a patient may have been given both a blood and urine test during a visit, or only one of these at a particular time.

%in the results discuss the strength and limaitaions of this apprach

The continuous time extension to the Baum Welsh algorithm, used the Kolmorgorox equations, which map the transition matrix of a discrete time model to CTBNs. Initial values for the each patients intensity matrix were obtained by using a naive estimation provided by counting the total number of transition pairs for the entire population, and estimating their probability of occurrence. Although not all patients were able to have model parameters output by the abstraction method, initial population level estimates allowed more observations to converge using the parameter estimation methods than the same naive initialization assumption at the patient level.

Using a 4-state multi-state Markov model, each patient's parameters are calculated using likelihood estimations based on the time and values in their observation sequences.  The parameter, or abstraction, used as input to clustering is the matrix $Q$, representing the instantaneous behavior of the process $X$, as an $n$x$n$ matrix.
The model is initialized using To learn the priors for each patients model, using BFGS.




\begin{table}[ht]
\caption{Input for model abstraction step}
\label{hepinput}
\vskip 0.3in
\begin{center}
\begin{tabular}{lcc}
\hline
data& state	& PLT	\\
\hline
19811111& 2&177 \\
19830720&2&182 \\
19830818&2&167 \\
\hline
\end{tabular}
\end{center}
\vskip -0.1in
\end{table}



\subsubsection{Inference}
\subsubsection{Model Comparison}
\subsection{Clustering}


\subsection{Validation}
\label{hep_results}
To validate the results of clustering temporal lab data for the hepatitis data set, we used grading and staging data from liver biopsies a gold standard.  We validate our results using the b-cubed metric and compare performance with results that of previous work using temporal PLT data alone~\cite{Hirano05}, and multivariate temporal data, including PLT-ALB data~\cite{ Hirano07a,Hirano07b} and work publishes in 2012, which extended the approach to PLT, ALB and ChE tests~\cite{Tsumoto12}.  These prior studies used a feature-based technique, ``\emph{trajectory mining}", and provide results that show that compared with conventional studies the method provides a more detailed classification of temporal trends.

I also compare results for alternative clustering algorithms for the semiparametric framework.  Typically, spectral methods are the nonparametric clustering technique of choice.  However, they have some limitations that are discussed in more detail in Chapter~\ref{5_cluster_alg}.  As an alternative, I compare results obtained with spectral clustering with that of Drichlet process Gaussian mixture models, an approach that has its foundation in density estimation and has more recently applied for clustering.  The motivation for applying nonparametric Bayesian for clinical data is described in Chapter\ref{6_extending}.

Lastly, to provide some context for the choice of previous work to focus on only those patients with HVC and no interferon therapy, one needs to be aware of affect on relevant biomarkers.  HVC was considered an nontreatable disease until interferon therapy.  Although beneficial for some patients, there can be serious side effects.  In terms of diabetes indicators, it can cause fluctuations.  Therefore, is can be helpful to distinguish the population of patients with no interferon treatment.

\subsubsection{Comparison of Bayesian and Spectral Methods}
First, we compare alternative nonparametric methods for the semiparametric clustering of temporal hepatitis data, specifically spectral clustering and Bayesian clustering.  Figure~\ref{allHep} visualizes the validation scores reported in Table~\ref{allHepTable}.

\begin{figure}[ht]
\vskip 0.2in
\begin{center}
\centerline{\includegraphics[width=\columnwidth]{fig/allHep.jpeg}}
\caption{B-cubed value for different nonparametric clustering methods and $k$ values for the hepatitis data set}
\label{allHep}
\end{center}
\vskip -0.2in
\end{figure}


\begin{table}[ht]
\caption{B-cubed value for spectral (SP-SC) and nonparametric Bayes (SP-B) clustering, all hepatitis patients.}
\label{allHepTable}
\vskip 0.15in
\begin{center}
\begin{small}
\begin{sc}
\begin{tabular}{lcccr}
\hline
\hline
method	& k	& P	& R	& B-Cubed \\
\hline
\hline
spectral methods & 6	& 0.38& 0.22& 0.28\\
spectral methods	         & 7	& 0.38& 0.21& 0.27\\
\textbf{Bayesian clustering}       & \textbf{4}	& \textbf{0.35}& \textbf{0.62}& \textbf{0.45}\\
Bayesian clustering	     & 5	& 0.35& 0.52& 0.42\\
Bayesian clustering	     & 6	& 0.36& 0.46& 0.40\\
\hline

SP-B	     & 7	& 0.36& 0.42& 0.39\\
\hline
\end{tabular}
\end{sc}
\end{small}
\end{center}
\vskip -0.1in
\end{table}



\subsubsection{Platelet Data}
Using \emph{only} platelet tests, Figure~\ref{plt_only} shows the results of semiparametric Bayesian clustering reported in Table~\ref{pltdatatable}.  The results are compared to techniques developed by Hirano et al.\cite{Hirano05}, and based on b-cubed values that were calculated using the cluster constitutions that were published in their paper, with the authors indicating that small clusters of $n \l 3$ omitted.

\begin{figure}[ht!]
\vskip 0.2in
\begin{center}
\centerline{\includegraphics[width=\columnwidth]{fig/plt_only.jpeg}}
\caption{Comparison of clustering methods using temporal PLT data}
\label{plt_only}
\end{center}
\vskip -0.2in
\end{figure}

%Bayesian clustering	hepC_noInf	plt	5	0.491029213	0.411852866	0.447969438
%Hirano 2007b	hepB	plt	8	0.256698688	0.277312663	0.308345178
%Hirano 2007b	hepC_noInf	plt	6	0.455202852	0.385447555	0.417431134
%Hirano 2007b	hepC_Inf	plt	11	0.386774601	0.229170177	0.28780893
%BC all 	    5	0.330556375	0.386385095	0.356297025	472	
%BC hepC_noInf	5	0.491029213	0.411852866	0.447969438	94	
%BC all         8	0.370831073	0.348019617	0.359063405	472	
%BC hepC_noInf  6	0.477650748	0.424699113	0.449621278	94	
%BC all 	    4   0.357703975	0.38512469	0.370908229	472	
\begin{table}[ht]
\caption{Precision, recall, and B-cubed scores for alternative systems using only temporal PLT data.}
\label{pltdatatable}
\vskip 0.15in
\begin{center}
\begin{tabular}{lccccr}
\hline
\hline
method	& sample &k	& P	& R	& B-Cubed \\
\hline
\hline
Bayesian clustering	& HVC no Tx & 5& 0.49& 0.41& 0.45 \\
Bayesian clustering	& HVC no Tx & 6& 0.48& 0.42& 0.45 \\
Bayesian clustering	& all & 4& 0.36& 0.39& 0.37 \\
Bayesian clustering	& all & 5& 0.33& 0.39& 0.36 \\
Bayesian clustering	& all & 8& 0.37& 0.35& 0.36 \\
Hirano 2005 & HVC no Tx   & 6& 0.46& 0.39& 0.42 \\
Hirano 2005 & HVC Tx   & 11& 0.39& 0.23& 0.29 \\
Hirano 2005 & HVB   & 8& 0.26& 0.28& 0.31 \\
		
\hline
\hline
\end{tabular}
\end{center}
\end{table}


\subsubsection{Hepatitis C, No Interferon Therapy}
\label{hep_results_noinf}
Figure~\ref{hepC_noinf_fig} shows the results of alternative clustering methods on the HCV population with no indication of interferon therapy.  Detailed scores appear in Table ~\ref{hepC_noInftable}.

\begin{figure}[ht!]
\vskip 0.2in
\begin{center}
\centerline{\includegraphics[width=\columnwidth]{fig/hepC_noInf.jpeg}}
\caption{Comparison of semiparametric clustering with a various benchmarks}
\label{hepC_noinf_fig}
\end{center}
\vskip -0.2in
\end{figure}

\begin{table}[ht!]
\caption{Validation scores for HVC patients, no interferon therapy}
\label{hepC_noInftable}
\vskip 0.15in
\begin{center}
\begin{tabular}{lcccr}
\hline
\hline
method	& k	& P	& R	& b-cubed \\
\hline
\hline
Hirano 2007	& 8& 0.60& 0.31& 0.41 \\
\textbf{Tsumoto 2012}	& \textbf{9}& \textbf{0.57}& \textbf{0.33}& \textbf{0.42} \\
\textbf{Bayesian clustering}	& \textbf{4}& \textbf{0.47}& \textbf{0.55}& \textbf{0.51 }\\
Bayesian clustering	        & 5& 0.50& 0.47& 0.48 \\
\textbf{Bayesian clustering}	        &\textbf{ 6}& \textbf{0.48}& \textbf{0.54}& \textbf{0.51} \\
\hline
\hline
\end{tabular}
\end{center}
\end{table}

 Semiparametric Bayesian clustering improves on trajectory mining methods using temporal PLT and ALB data, which showed a 42\% b-cube value~\ref{Hirano07a,Hirano07b} and work publishes in 2012, which extended the approach to PLT, ALB and ChE data, and reports a cluster composition that corresponds with a 42\% b-cubed value ~\cite{Tsumoto12}.

%UPDATE the table
%Bayesian clustering	hepC_noInf	alb-plt-zzt	4	0.466122023	0.554483985	0.506477906
%Bayesian clustering	hepC_noInf	alb-plt-zzt	5	0.495159527	0.469947497	0.482224198
%Bayesian clustering	hepC_noInf	alb-plt-zzt	6	0.476378599	0.538087179	0.505356065
%Bayesian clustering (PLT)	hepC_noInf	plt	5	0.491029213	0.411852866	0.447969438
%Hirano 2007b (PLT)	hepC_noInf	plt	6	0.455202852	0.385447555	0.417431134
%Tsumoto 2012	hepC_noInf	alb-che-plt	9	0.567928605	0.328279479	0.416062542
%Hirano 2007a	hepC_noInf	plt-alb	8	0.5956	0.3097	0.4075


\subsubsection{Semiparametric Bayesian Clustering}
To more thoroughly assess the performance of semiparametric clustering using gold standard results, Figure~\ref{spBayes} shows results for all patients, and the subset of patients with HVC, using temporal data from temporal PLT, ALB and ZZT data. The precision is on the $x$-axis and the recall on the $y$-axis.  The b-cubed values range from dark to light, with lighter values indicating higher scores.

Table~\ref{spbdatatable} reports the results shown in~\ref{spBayes}.  Notably, the best assignment for HVC patients with no interferon therapy reported an 51\% B-cubed for $k=4$ with results for $k=6$ close behind.  For clustering all patients, the results show a 45\% B-cubed value for $k=4$.

\begin{figure}[ht]
\vskip 0.2in
\begin{center}
\centerline{\includegraphics[width=\columnwidth]{fig/spBayes.jpeg}}
\caption{Comparison of semiparametric Bayesian clustering by patient population}
\label{spBayes}
\end{center}
\vskip -0.2in
\end{figure}

\begin{table}[ht]
\caption{Precision, recall, and B-cubed scores for semiparametric Bayesian clustering.}
\label{spbdatatable}
\vskip 0.15in
\begin{center}
\begin{tabular}{lcccr}
\hline
\hline
method	& k	& P	& R	& B-Cubed \\
\hline
\hline
All Pts	& 4& 0.35& 0.62& 0.45 \\
All Pts & 5& 0.35& 0.52& 0.42 \\
All Pts & 6& 0.36& 0.46& 0.40 \\
All Pts & 7& 0.36& 0.42& 0.39 \\
		
\hline
\hline
HVC no Tx	& 4& 0.47& 0.55& 0.51 \\
HVC no Tx & 5& 0.50& 0.47& 0.48 \\
HVC no Tx & 6& 0.48& 0.54& 0.51 \\

\hline
\end{tabular}
\end{center}
\end{table}

%Bayesian clustering	all	alb-plt-zzt	4	0.351527266	0.622462748	0.449311851
%Bayesian clustering	all	alb-plt-zzt	5	0.352476136	0.518519429	0.419670851
%Bayesian clustering	all	alb-plt-zzt	6	0.356380466	0.458481777	0.401034532
%Bayesian clustering	all	alb-plt-zzt	7	0.358414504	0.420788916	0.387105207
%Bayesian clustering	hepC_noInf	alb-plt-zzt	4	0.466122023	0.554483985	0.506477906
%Bayesian clustering	hepC_noInf	alb-plt-zzt	5	0.495159527	0.469947497	0.482224198
%Bayesian clustering	hepC_noInf	alb-plt-zzt	6	0.476378599	0.538087179	0.505356065





%\subsubsection{Discrete versus Continuous Time Abstraction}
%To validate the results of clustering platelet count values for the hepatitis data set, we used grading and staging data from liver biopsies. Figure~\ref{biopsy} show the result from the clustering with the highest highest purity (0.61\%) and obtained using continuous-time HMM abstraction paired with non-parametric Bayesian clustering.  Cluster purity obtained using spectral clustering, was notably lower, reporting a high of $0.40$.  Cluster membership is visualized in relation to highest biopsy grading, and reported by percent of the total for each grade ($n=468$).  A grading of four indicates cirrhosis of the liver or advance scarring. In addition to grading, biopsy activity is commonly used to stage liver disease and visualized in Figure 2. using the fill variation for each pie.

%\begin{figure}[h]
%\centering
%\includegraphics[width=85mm]{fig/hep_2.jpg}
%\caption{Hepatitis Clusters by Biopsy Staging and Activity}
%\label{biopsy}
%\end{figure}

%External data for validating our glucose data set clustering was not available and intrinsic measures based on cluster silhouettes were used to assess cluster quality.  The results of CT-HMM abstraction paired with non-parametric Batesian clustering paired is reported in terms of average silhouette is shown in~\ref{table:sil}.  Spectral clustering was also paired with CT-HMMs.  Although the average silhouette value was overall lower (0.10), one large cluster had a average silhouette higher that that on the max for non-parametric Bayesian clustering.

%\begin{table}[h]
%\caption{Cluster Silhouettes}
%\centering
%\begin{tabular}{|l|c|c|}
%  \hline
  % after \\: \hline or \cline{col1-col2} \cline{col3-col4} ...
%   Cluster& Members & Average Silhouette  \\
%   \hline
%  $C_1$ & 153 & -0.04956 \\
%  $C_2$ & 269 & 0.9416  \\
%  $C_3$ & 132 & 0.1151  \\
%  $C_4$ & 114 & -0.3712 \\
%  $C_5$ & 337 & -0.5460 \\
%  \hline
%\end{tabular}
%\label{table:sil} % is used to refer this table in the text
%\end{table}

%Compared with discrete-time HMM abstraction for the same data set, in previous work (Tamang 2011) we reported over 80 percent of patients with a good (0.60 or above) silhouettes value.  In comparison, continuous-time HMM abstraction report just over half of patients with a good silhouette value (54 percent).
%\subsubsection{Non-parametric Clustering Alternatives}



\subsection{Discussion}
In addition to evaluation of semiparametric Bayesian clustering with a gold standard, the experiments with the hepatitis data set are designed to compare the approach with an alternative spectral clustering step and previous temporal clustering results reported in the literature.  Here, I discuss the key findings based on external metrics and their clinical significance.  Also, I show a visualization technique to aid the interpretation of clustering results that is based on average model parameters for each discovered cluster, or patient group.

\subsubsection{Spectral Clustering Versus Nonparametric Bayesian Clustering}

 By comparing alternative clustering steps for the semiparametric framework, these results indicate a benefit to \emph{Drichlet process Gaussian mixture modeling}, an nonparametric Bayesian clustering, instead of spectral method.  Due to popularity of spectral methods, and reputation for high performance, my expectation was comparable performance.  I suspected the key benefit would be a relaxation of the requirement to indicate $k$ $a$ $priori$ and not a notable improvement in overall performance.

 One explanation for the improved performance is the distribution of the class labels, which is Gaussian.  Spectral methods methods attempt to balance the size of the clusters while minimizing the interaction between dissimilar points, and can bias results towards clusters of equal size~\ref{WhiteS05}.  Many diseases, and disease-related states show Gaussian population distributions, and our results suggest that spectral clustering may not be the best choice for modeling patient populations.

\subsubsection{Comparison with State-of-the-Art Methods}
To compare the results of semiparametric Bayesian clustering with previous work, I reconstructed a univariate and multivariate experiments and compared them with published results~\cite{Hirano05,Hirano07a,Hirano07b,Tsumoto12}.  I show that it out performs alternative methods for range of scenarios, including univariate and multivariate analysis, and different hepatitis disease types.

Using the best temporal indicator, PLT, results for univariate temporal clustering were small (42\%~\cite{Hirano05} to 45\%).  However it is important to note that in comparison experiment, small clusters where $N$<$3$ were not provided in the membership table.  How generous the score calculated from their cluster constitution table is unclear.

For multivariate temporal data, clustering performance increases over a 20\% relative improvement using combined ALB-PLT-ZTT tab results, reporting a 51\% b-cube score.  Cluster constitutions reported in the literature based on trajectory mining, reported scores between 41-42\% and were based on the temporal modeling of features from combined ALB-PLT~\cite{Hirano07a,Hirano07b} and ALP-PLT-ChE lab data~\cite{Tsumoto12}.

Again, some of the comparison estimates are generous.  The clusters where $N$ < $2$ for the temporal ALP-PLT data results were not provided in the membership table, or could be determined by another reported statistic.  Inclusion of even one of the missing clusters would have reduced the completeness constraint, and lowered the value of the validation metric.

\subsubsection{Clinical Context and Relevance}
The goal of temporal mining task was to discover patient groups that are useful or meaningful to:
 \begin{itemize}
   \item discriminate between those that progress to more advances stages such as cirrhosis or hepatocarcinoma, and
   \item determine if less invasive diagnostic procedures can replace liver biopsy.
 \end{itemize}

In addition to cluster membership, semiparametric Bayesian clustering provides us with additional information for learning group level properties from members' trajectories.  The subpopulation characteristics can be used to assign a unseen patient to one of the discovered clusters.   More specifically, each patient's model parameters characterizes their rate of change from of the disease states to all others.  In Figure\cite{fig:hepq} we show averages for each cluster's characteristic model, based on a three state (low,normal,high) $k=4$ PLT based clustering results.

%1       1  0   4.649329  -9.866057 -41.078529  0 -18.21271 -9.058329 11.266929
%2       2  0  -4.832645 -19.860027  -6.281500  0 -16.76764 -8.736268 -1.768516
%3       3  0 -34.752063 -23.149088   5.142881  0 -14.15912  1.595881 -5.364187
%4       4  0  15.124626   8.949241 -59.671174  0 -27.21240 -8.355130 17.715889


\begin{figure}
\[Q_1= \left( \begin{array}{ccc}
0 & 4.65 & -9.87 \\
-41.08 & 0 & -18.21 \\
-9.06 & 11.27 & 0
\end{array} \right)
%
Q_2= \left( \begin{array}{ccc}
0 & -4.83 & -19.86 \\
-6.28 & 0 & -16.77 \\
-8.74 & -1.76 & 0
\end{array} \right)
\]
\[ Q_3= \left( \begin{array}{ccc}
0 & -34.75 & -23.15 \\
5.14 & 0 & -14.16 \\
1.60 & -5.36 & 0
\end{array} \right)
%
Q_4= \left( \begin{array}{ccc}
0 & 15.12& 8.95 \\
-59.67 & 0 & -27.21 \\
-8.36 & 17.72 & 0
\end{array} \right)
\]
\caption{Intensity Matrices for 3-state, 4 cluster hepatitis model}
\label{fig:hepq}
\end{figure}

The cluster composition is shown by fibrosis stage, the gold standard class used for external validation, appears in Table\ref{hepassignments}.  Fibrosis stages are labeled F0 through F4, with F4 indicating end-stage liver disease.


\begin{table}[ht]
\caption{Cluster composition by fibrosis stage}
\label{hepassignments}
\vskip 0.15in
\begin{center}
\begin{tabular}{lcccr}
\hline
\hline
	& $F0,F1$ &$F2$	& $F3$	& $F4$ \\
\hline
\hline
$k_1$ & 5& 2& - & - \\
$k_2$ & 17& 11& 6& 10 \\
$k_3$ & 6& 1& 3& 6\\
$k_4$ & 25& 1& -& 1\\
		
\hline
\hline
\end{tabular}
\end{center}
\end{table}

\subsubsection{Risk of End-stage Liver Disease}
\label{risk}
To \emph{stratify discovered groups by risk types}, I examine the proportion of class labels within each.  Figure~\ref{hep_2} shows the cluster composition as a percent of total records for each class.  The size of each circle is proportional to the total count for each Matavir score, or class label. Also, it shows the biopsy activity, HAI, with the darkest color indicating the highest activity, and which some consider a better indicator for liver fibrosis than the Matavir score.

\begin{figure}[t]
\vskip 0.2in
\begin{center}
\centerline{\includegraphics[width=\columnwidth]{fig/hep_2.jpg}}
\caption{Metavir biopsy grade and Hepatitis Activity Index (HAI) by cluster as a percent of total records for each class}
\label{hep_2}
\end{center}
\vskip -0.2in
\end{figure}

Based on Figure~\ref{hep_2}, we can broadly rank clusters by increased risk for progressing to end-stage liver disease: $c_4 < c_1 < c_2 < c_3$.  In in $c_1$, the presence of the largest circle at the lowest grade, F0, and a steady decrease in size as biopsy grades increase, indicates the highest composition of patients at low risk.  Consistent with this conclusion, relative to other clusters, $c_1$ patients have a lower activity with only a small fraction showing high biopsy activity.  Similar to $c_1$, $c_4$ also shows the presence of patients with lower biopsy grades, and low activity.  However, it does not show the a larger proportion of these patients at lower biopsy grades, as in $c_1$, suggesting that members of $c_4$ progress to end-stage liver disease more often that those patients in $c_1$, but based on cluster composition, less often that patients in $c_2$, or $c_3$.  Patients in $c_2$ are the most likely to progress to higher fibrosis stages, showing the highest proportion of patients with F3 and F4 scores, that decrease as the fibrosis grade gets lower.  A similar trend is observed in $c_3$ but is not as dramatic.

\begin{figure}[ht!]
\vskip 0.2in
\begin{center}
\centerline{\includegraphics[width=\columnwidth]{fig/hepc_qmatrix.jpeg}}
\caption{Comparison of semiparametric clustering with a reported benchmark}
\label{hepCqmatrix}
\end{center}
\vskip -0.2in
\end{figure}

\subsubsection{Hepatitis Disease Trajectory}
  The irregular sampling of disease related indicators, and the long progression times are obstacles to providing more detailed mechanism for chnonic diseases trajectory that are not well understood, such as hepatitis.  One benefit of model-based temporal abstraction with Bayesian network is that unlike feature-based methods that aim to characterize salient properties of the observed signal, they provide a direct representation of the underling process or phenomena they model.

  For non-canonical time series data, continuous-time Bayesian networks (CTBNs) can provide parameters for patient and population level models that can be used to provide insight into chronic disease progression, specifically, the instantaneous rate of change between of this disease states.  The added benefit of clustering is the division of a patient sample into subpopulations that can correspond with meaningful groups such as risk types, providing the ability to examine hepatitis trajectories at a finer-level of detail that is useful for characterizing patients.

  Figure\cite{fig:hepq} shows a characteristic intensity matrix for each of the clusters. To facilitate their interpretation, and compare clusters in terms of instantaneous risk factors, Figure~\ref{hepCqmatrix} visualizes the each of the intensity matrices.  Each cluster is represented by a color along the $y$-axis, and the $x$-axis represents the the number of standard deviations each cluster's dataum is relative the mean for all of the $q_{ij}$.  Each of the nine plots in the 3x3 grid that correspond with the entries of the $Q$ matrix, and represent the transitions from low, normal and high disease states.

Based on our assessment of risk types in Section~\ref{risk} that broadly qualified the clusters by increased risk of end-stage liver disease as $c_4 < c_1 < c_2 < c_3$, we can describe properties of patient sub-populations in terms of instantaneous risk, and relative to each other.  For example, if a patient is currently in a normal state, how likely are they to progress to lower (unhealthy) state?  Figure\cite{fig:hepq} shows us the model parameters for each cluster, and indicated the precise value of the instantaneous risk.

To compare sub-populations relative to each other, the $q_norm,low$ entry in Figure~\ref{hepCqmatrix} shows us that patients in $c_1$ and $c_4$, which correspond with those at less risk of fibrosis, are more likely to remain in the normal state, and patients in  $c_1$ and $c_4$, the patient groups with the highest fibrosis risk, are more likely to transition to poorer health based on the population average. Also, it shows that $c_4$ represents patients are mostly likely to remain in a normal state and $c_3$ patients are the most likely to transition to a poorer health states.  Notably, this interpretation is consistent with our designation for risk types, and their rank from lowest to highest risk, and can be made for other key model transitions.

\subsubsection{Actionable Findings}
One important finding that is not adequately captured by standard evaluation metrics that is relevant to our problem is related to the increased importance of lowest fibrosis class, composed of patients labeled F0 or F1 in the gold standard.  Specifically, liver biopsy is an invasive procedure that can put patients at risk of procedural complications, and costs hundreds of dollars.  The ability to reduce the number of unnecessary biopsies is currently an active area of research.

The lowest risk cluster generated by our procedure consistently produced one cluster of high purity for F0 and F1 patients.  Although this cluster was not maximal for this population, it can be used to identify patients for which biopsy is likely unnecessary.  For example, 93\% of patients in $k_4$  (see Table\ref{hepassignments}) had unnecessary biopsies, and represent about a quarter of the patient sample.





%cluster assignments appear in the NYU slides 
