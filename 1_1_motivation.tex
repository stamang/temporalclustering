\section{Motivation}
The availability of clinical data repositories presents new opportunities for improving health care quality and reducing costs.  However, the secondary function of these data sources as research tools presents challenges for longitudinal data analysis.  To facilitate the meaningful use of abundant temporal EHR data, I develop a probabilistic \emph{temporal clustering method} that can assist in the preprocessing, exploration, and discovery of new knowledge from these noisy, heterogenous, fragmented data collections.

Clustering is a pervasive and natural human activity that affects knowledge representation and discovery~\cite{Guyon}.  Typically, we use it to group similar objects together, and establish criteria that are useful for their definition.  Computational clustering algorithms can help reveal inherent group structures, or clusters, among observations without requiring the use of class labels for learning.  For large data sets that are high dimensional, and where the distinguishing characteristics for establishing class boundaries are unknown, this application is particularly useful for exploratory analysis, and can enable a systematic examination of a data set.  In addition, clustering can be used as a preprocessing step with the aim of enriching the quality of a data sample by filtering noisy or less informative examples.

For phenomena that evolve over time, the magnitude and direction of changes, and when changes occur can provide critical context for reasoning.  Not only are there resource issues associated with the increased size of a temporal data set, but modeling choices related to the representation of temporal granularity and sequential dependencies must be considered.  Also, secondary sources of temporal data are especially challenging.  Many contain measurement sequences that are irregular in length, and subject to a variety of sampling schemes, resulting in irregular time intervals between observations, and data that is not missing at random.

%Although temporal data provides critical context for many real-world reasoning tasks, incorporating the temporal dimension into an analysis can present methodological challenges.  Traditional methods from statistics are limited in their ability to scale to data sets with thousands of time points and noisy secondary data sources, and data mining approaches have primarily focused on static data sets.  However, few real-world data sets are by nature static, or are designed to measure stationary phenomena; rather, they are dynamic and contain information about values that change over time.

 %A variety of learning algorithms, have been developed to address the limitations of traditional approaches for modeling real-world time series sequences that are large-scale, subject to noise, and serve a secondary purpose as a research instrument.

 A variety of learning algorithms, have been developed to address the limitations of traditional approaches for modeling real-world time series sequences.  Regardless of the task, the first step involves \emph{abstraction}, a process of transforming the raw data in to a high-level representation that facilitates description and comparison among sequences, typically to retain only information which is relevant for a particular purpose.  Of the main methods, model-based abstraction is the only technique that provides a direct correspondence with the description of the underlying data generating process, and provides a principled way to transform variable length observation sequences into a uniform number of descriptive parameters, facilitating its use for additional analysis and interpretation.

To cluster temporal sequences, the \emph{semiparametric clustering} framework described by Jebara et al.~\cite{Jebara} pairs structural and parametric assumptions such as Markov properties for discrete-time abstraction, with nonparametric spectral methods to cluster the abstractions, priors of the model parameters for each sequence, instead of the raw time series sequences.  In addition to demonstrating state-of-the-art performance on a variety of temporal data sets, their experiments show clear benefits of hybrid methods instead of fully parametric, or fully non-parametric methods.

In this thesis, I extend the semiparametric clustering framework to address common challenges to modeling temporal EHR data.  Although discrete-time assumptions provide useful simplifying assumptions for modeling many temporal data sets, Bayesian networks~\cite{Dean89} and their variants are less appropriate for processes that that are observed by arbitrary sampling schemes, such as self-selection, or that may evolve on a non-linear trajectory.  Also, spectral clustering methods assume the number of clusters, $k$, is fixed.  For patient data, where the number of groups can vary based on the sample, a more flexible approach that allows $k$ to be expressed as a function of the size and complexity of the data is more desirable.

To more directly represent patient data related to chronic disease progression, the specific models I develop for abstraction are based on finite state \emph{continuous-time (CT) Markov processes}.    Specifically, I use CT Bayesian networks (CTBNs)~\cite{Nodelman02}, incorporating theoretical aspects of multi-state Markov models (MSMs)~\cite{Jackson10} that are used form disease modeling in epidemiology and biostatistics.  MSMs share the same theoretical foundations as CTBNs and are based on domain knowledge that many chronic diseases have a natural interpretation that results in a staged progression of disease related states. In addition, I extend applications of clustering to the nonparametric Bayesian setting, allowing for a clustering model that is more expressive, and no longer requires that $k$ is established by a heuristic $a$ $priori$.

%When there are no natural time slices, CT Bayesian networks (DBNs)~\cite{Nodelman02} more directly reflect sequential dependencies, and in contrast to discrete-time models do not force the representation of missing data.  Notably, a class of DBNs known as multi-state Markov models (MSMs)~\cite{Jackson10} in biostatistics.  Traditionally, MSMs are used to examine population based disease dynamics from smaller longitudional data sets.  They have not been applied for comparisons among patients, or as the basis for temporal clustering.


%To more directly represent time, I extend temporal abstraction to an class of continuous-time Bayesian networks (CTBNs)~\cite{Nodelman02}, which share with multi-state Markov models (MSMs)~\cite{Jackson10}.  MSMs are used by epidemiologists to model population disease dynamics in biostatistics and share a foundation in finite state \emph{continuous-time (CT) Markov processes}.    Specifically, In addition, I extend applications of clustering to the nonparametric Bayesian setting, allowing for a clustering model that is more expressive and no longer requires that $k$ is known $a priori$.

%Traditionally, MSMs are used to examine population based disease dynamics from smaller longitudional data sets.  They have not been applied for comparisons among patients, or as the basis for temporal clustering.

%The experiments I conduct aims to cluster patients of similar health status together in the same cluster, and to represent distinct qualitative groups in separate clusters, with each cluster of maximal size.

%Although this framework is appropriate for many types of data sets, the types of models used for abstraction, Bayesian networks~\cite{Dean89} and their variants, make discrete-time assumptions that are less appropriate for data that is not missing at random, and EHR data can be subject to arbitrary sampling schemes, and patient self-selection is common.  Also, spectral clustering has been the consistent choice for the nonparametric clustering step of the semiparametric framework, and requires that the number of clusters, $k$, is fixed.  For patient data, where the number of groups can vary based on the sample a more flexible approach that allows $k$ to be expressed as a function of the size and complexity of the data is more desirable.

%To more directly represent time, the specific models I develop for temporal abstraction are based on finite state \emph{continuous-time (CT) Markov processes}.    Specifically, I extend temporal abstraction to an class of CT Bayesian networks (DBNs)~\cite{Nodelman02} known as multi-state Markov models (MSMs)~\cite{Jackson10} in biostatistics.  Traditionally, MSMs are used to examine population based disease dynamics from smaller longitudional data sets.  They have not been applied for comparisons among patients, or as the basis for temporal clustering.

%Rather than use the sequences of values directly, we build probabilistic models, abstractions, of these sequences using continuous-time models.

%In this thesis, using longitudinal data extracted from EHRs to model chronic dynamics, I extend the framework of semiparametric temporal clustering to address the some of the challenges posed by clinical data, and describe a new exploratory analysis method for learning patient and population level chronic disease dynamics.  Specifically, I extend the framework of semiparametric clustering to continuous-time Bayesian networks (CTBNs)~\cite{Nodelman02} for model-based abstraction.  When there are no natural time slices, CTBNs more directly reflect sequential dependencies, and in contrast to discrete-time models do not force the representation of missing data. In addition, I extend applications of clustering to the nonparametric Bayesian setting, allowing for a clustering model that is more expressive and no longer requires that $k$ is known $a priori$.



%Clinically significant work applying exploratory techniques to clinical data has been demonstrated~\cite{Saria09, Marlin12}, but it has focused on monitoring patients in the critical care setting, where data is sampled at a high frequency.  In contrast, modeling the evolution of a patient's chronic disease trajectory may require data that has been sampled for a duration of months or years, and in EHR collections, the assumption that data is missing at random is less likely to be true.


%The key benefits of this abstraction approach is a principled way to represent variable length raw data as a uniform length vector.  However, they require that time is discretized, and represented as series of steps that are equal to the smallest time granularity in any observed sequence.  Although discrete-time models are suitable in many cases, there are two key limitations that  relevant to EHR data. First, if the underlying health phenomena progresses in individuals at different rates, one granularity must be used to express time steps for the entire system. Second, when data is unavailable, intervening time slices must still be represented.

%When there are no natural time slices, continuous-time BNs (CTBNs) (Nodelman et al., 2003) can be used to more directly reflect sequential dependencies, and avoid discretizing the time intervals.  Notably, a domain specific instance of CTBNs, Multi-state models (MSMs), are used in epidemiology and biostatistics as Multi-state models (MSMs) for modeling population level disease dynamics.  Our work builds upon these theoretical foundations in CTBNs and MSMs to develop a continuous-time abstraction method for longitudinal EHR data.

Lastly, the reasons described above motivate this work but it is important to note the following.  Although unsupervised learning methods have many successful applications, one substantial obstacle to their practical adoption is the difficulty of interpreting results.  Many clustering metrics are based on a mathematical interpretation of clustering as a partitioning problem that is independent of problem context.  However, it is clear the purpose of clustering, and the context of its end-use are crucial aspects to consider when determine if results are useful or meaningful~\cite{Guyon}. For this reason, I evaluate my method using established validation metrics from the literature, and develop visualization tools to help communicate and examine the clinical context and significance of my work.

%Bayesian Nonparametrics is a rapidly growing subfield of statistics and machine learning that provides a framework for creating complex statistical models that are both expressive and tractable.

%some beneficial structural and parametric assumptions such as Markov properties and hidden state variables which are useful for clustering

%For the purpose of inference, the goal of temporal abstraction is to provide a concise, high-level description of a sequential process that facilitates description and comparison among sequences, while simultaneously preserving the information contained in the raw data.

%can provide insight into the nature of dynamic processes that are not well understood.  For scientific inquiry, they help automate the systematic examination of an entire collection, and can be used to generate new and important hypotheses that may not have been identified by a human expert.   Also, they can be used as a preprocessing step with the aim of enriching the quality of a data sample by filtering noisy or less informative examples.

%Although unsupervised learning methods are the focus of considerable work in machine learning, extending their use for sequential data remains challenging.

%Despite noted limitations by researchers, clustering methods have been used effectively as an exploratory analysis step for driving scientific inquiry towards more targeted hypotheses, or as a preprocessing step to enrich the quality of a data sample by filtering noisy or less informative examples.  As databases evolve over time, growing in number, size and scale, and more resources are available to efficiently processing data, methods for temporal data modeling and analysis are of increasing importance.  %In the context of inquiry,  exploratory methods methods can enable exploratory analysis, and be used to generate new and important hypotheses that may not have been identified by a human expert.
%Also, they help automate the systematic examination of an entire collection instead of a preselected sample, reducing the impact of experimenter and other biases that have been associated with false findings~\cite{Ioannidis05}, and can be used to generate new and important hypotheses that may not have been identified by a human expert.

%temporal analysis
%Time provides critical context for many real-world reasoning tasks and temporal clustering can provide insight into the nature of dynamic processes that are not well understood.  For scientific inquiry, they help automate the systematic examination of an entire collection, and can be used to generate new and important hypotheses that may not have been identified by a human expert.   Also, they can be used as a preprocessing step with the aim of enriching the quality of a data sample by filtering noisy or less informative examples.

%Existing data mining approaches have demonstrated their ability to reveal unseen patterns and summarize data in meaningful ways, but they have primarily focused on static data sets, where the assumption that data is independently drawn and distributed (i.d.d.) is less problematic.  However, few real-world data sets are by nature static, or are designed to measure stationary phenomena; rather, they are dynamic.   Not only is the i.i.d. assumption more problematic for temporal learning, real-world data sets often provide measurement sequences that are irregular in length, arbitrarily sampled and may contain data that is not missing at random (MAR).  Together these collective conditions present methodological callenges for existing data mining methods, and more traditional statistical methods.

%temporal clsutering
%Regardless of the temporal learning method, abstraction is almost always
%Abstraction is the process by transforming a concept or an observable phenomenon in to a more succinct form, typically to retain only information which is relevant for a particular purpose. For the purpose of inference, the goal of temporal abstraction is to provide a concise, high-level description of a sequential process that facilitates description and comparison among sequences, while simultaneously preserving the information contained in the raw data.

%A range of methods have beed developed to address the limitations of traditional analysis methods for modeling time.  They can be broadly categorized as subsequence, feature and model based, and primarily aim to transform the raw data into a more concise, high-level description of a sequential process that facilitates description and comparison among sequences.   In contrast to subsequence based methods, model based approaches can be used to abstract whole sequences, and facilitate the comparison of uneven length sequences.  Also, they do not require the overhead of a feature engineering process, which aims to extract information in order to characterize the shapes that correspond with change along the temporal trajectory.

%What has been developed in the data mining and machine learning community for temporal modeling are two main approaches: feature engineering and model based abstraction.  There are many successful feature based temporal mining algorithms, but they are task specific.  Feature-based approaches extract information in order to characterize the shapes that correspond with change along the duration of the temporal trajectory.  Also, they don't have a direct correspondence with the description of the phenomena or underlying process more generally.

%Bayesian Nonparametrics is a rapidly growing subfield of statistics and machine learning that provides a framework for creating complex statistical models that are both expressive and tractable
