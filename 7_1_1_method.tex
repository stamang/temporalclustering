\subsection{Data Description}
The hepatitis data set used for my experiments consists of blood inspection and urinalysis laboratory data that was provided by the Chiba University Hospital in Japan, and was used for the ECML/PKDD-2004 and 2005 Discovery Challenges.  It consists of test values for 771 patients with hepatitis type B or C, and spans the years 1982 through 2001.

For each patient, the data set includes demographics, the pathological classification of the disease, the date of each biopsy, the result of each biopsy, and blood test and urinalysis results.  For patients with hepatitis C, there is an additional indication for interferon therapy, which is used to treat the disease and can effect the values of indicators for hepatocyte inflammation.  Although the temporal data contain the results of 983 types of examinations, we use the key indicators that are noted in the literature, and those indicators that showed diagnostic relevance for the machine learning challenges.

At the time of the challenge, medical research reported platelet count values (PLT) were correlated with fibrosis score at the time of biopsy, but cross-sectional analysis provided limited predictive power.  It was hypothesized that analyzing the trend for each patient over time could provide additional information for categorizing hepatitis patients by disease-related risk types.  However, temporal analysis of the PLT data was rarely performed and limited by difficulties in time series comparison, irregular sampling intervals, and variable sequence lengths~\cite{Shoji05}.

There were several goals posed by the ECML/PKDD shared task.  The specific challenge that relates to this work was determining the value of longitudinal laboratory exam data for assessing liver fibrosis, and to better understand the temporal patterns that correspond with the results of biopsy grading and staging.  Specifically, \emph{are there temporal patterns that can be detected from lab data to help distinguish patients that progress to end stage liver disease and those that do not?}

Although what was mainly developed consisted of classifiers, clustering applications for this data set appear in the literature.  Most notably the work of X et al. in 2007, and I used their results as a benchmark to evaluate performance.  Using PLT, ChE and ALB lab tests, one system~\cite{Hirano05,Hirano07a, Hirano07b,Tsumoto12} demonstrated that medically relevant time series features associated with the progression of liver fibrosis could be learned from clustering methods.  Other lab tests that were reported by challenge participants as informative for predicting fibrosis stage included: zinc turbidity test (ZTT), albumin (ALB), bilirubin D-BIL and CHE(cite task overview paper). 

\subsection{Abstraction}
\subsubsection{CTBN Model Structure}
A CTBN was used to abstract temporal information and is shown in Figure~\ref{msmgraph}.  The lab results for each patient over time were mapped to corresponding \emph{low}(L), \emph{normal}(N) and \emph{high}(H) values based on thresholds identified by clinical experts, and used as states in the model.  Table~\ref{hepxtab} shows the probability of each transition type based on the frequencies all patient transitions for platelet test values.  For example, based on reported transitions, patients reporting a high platelet count show the following probabilities for their next next observation period: high 92\%, normal 83\% and low is less than 1\%.
\emph{can we provide that averages for all patients here?}

\begin{figure}
\begin{center}
\begin{tikzpicture}[->,>=stealth',shorten >=1pt,auto,node distance=4cm,
  thick,main node/.style={circle,fill=blue!20,draw,font=\sffamily\Large\bfseries}]

  \node[main node] (1) {L};
  \node[main node] (2) [below left of=1] {N};
  \node[main node] (4) [below right of=1] {H};

  \path
    (1) edge node [left] {$q_{L,H}$} (4)
        edge [bend right] node[left] {$q_{L,N}$} (2)
    (2) edge node [right] {$q_{N,L}$} (1)
    	edge [bend right] node[below] {$q_{N,H}$} (4)
    (4) edge node [above] {$q_{H,N}$} (2)
        edge [bend right] node[right] {$q_{H,L}$} (1);
\end{tikzpicture}
\end{center}
\caption{3-state MSM}
\label{msmgraph}
\end{figure}

\begin{table}[ht]
\caption{Input for model abstraction step}
\label{hepxtab}
\vskip 0.15in
\begin{center}
\begin{tabular}{ l | c | c |c}
 $q_{i,j}$      & H & N & L  \\
\hline
H  & 0.915 & 0.083 & 0.001  \\\hline
N  & 0.035 &0.961 &0.004 \\\hline
L   & 0.025 &0.224 &0.751  \\
\end{tabular}
\end{center}
\vskip -0.1in
\end{table}

\subsubsection{Model Variables}
Based on an initial run using six lab tests that were selected due to known associations with liver decline, three, the PLT, ALB, and ZZT tests, resulted in good cluster assignments, with PLT performing the best based on the b-cubed metric~\cite{Bagga}, which is discussed in more detail in Section X, and is the harmonic mean of the average pointwise precision and recall for the cluster assignments compared with a gold standard.

To compare our results with previously published results I cluster patients based on PLT data alone, and in the multivariate lab test data.  To select additional indicators that are most useful, we generated preliminary clustering results using the five indicators that had been reported to have predictive qualities by unsupervised or supervised learning methods.  Using the variables that produced comparable results with that of PLT only, we calculated the mutual information between clusters assignments, and excluded ChE and D-BIL on this basis.  For multivariate temporal clustering of temporal ALB and PLT  data imporved results were achieved compared with results based on PLT alone, and ZZT made a minor improvement upon the results of ALB-PLT based clustering.

\subsubsection{Learning}
Multiple variables are combined for clustering, but the temporal data for each is abstracted separately.  Although the language of CTBNs allows for covariances, they must appear be observed concurrently, and there are numerous time days where the data does not appear for all three.  For example, a patient may have been given both a blood and urine test during a visit, or only one of these at a particular time.

%in the results discuss the strength and limaitaions of this apprach

The continuous time extension to the Baum Welsh algorithm, used the Kolmorgorox equations, which map the transition matrix of a discrete time model to CTBNs. Initial values for the each patients intensity matrix were obtained by using a naive estimation provided by counting the total number of transition pairs for the entire population, and estimating their probability of occurrence. Although not all patients were able to have model parameters output by the abstraction method, initial population level estimates allowed more observations to converge using the parameter estimation methods than the same naive initialization assumption at the patient level.

Using a 4-state multi-state Markov model, each patient's parameters are calculated using likelihood estimations based on the time and values in their observation sequences.  The parameter, or abstraction, used as input to clustering is the matrix $Q$, representing the instantaneous behavior of the process $X$, as an $n$x$n$ matrix.
The model is initialized using To learn the priors for each patients model, using BFGS.




\begin{table}[ht]
\caption{Input for model abstraction step}
\label{hepinput}
\vskip 0.3in
\begin{center}
\begin{tabular}{lcc}
\hline
data& state	& PLT	\\
\hline
19811111& 2&177 \\
19830720&2&182 \\
19830818&2&167 \\
\hline
\end{tabular}
\end{center}
\vskip -0.1in
\end{table}



\subsubsection{Inference}
\subsubsection{Model Comparison}
\subsection{Clustering}

